
% Default to the notebook output style

    


% Inherit from the specified cell style.




    
\documentclass[11pt]{article}

    
    
    \usepackage[T1]{fontenc}
    % Nicer default font (+ math font) than Computer Modern for most use cases
    \usepackage{mathpazo}

    % Basic figure setup, for now with no caption control since it's done
    % automatically by Pandoc (which extracts ![](path) syntax from Markdown).
    \usepackage{graphicx}
    % We will generate all images so they have a width \maxwidth. This means
    % that they will get their normal width if they fit onto the page, but
    % are scaled down if they would overflow the margins.
    \makeatletter
    \def\maxwidth{\ifdim\Gin@nat@width>\linewidth\linewidth
    \else\Gin@nat@width\fi}
    \makeatother
    \let\Oldincludegraphics\includegraphics
    % Set max figure width to be 80% of text width, for now hardcoded.
    \renewcommand{\includegraphics}[1]{\Oldincludegraphics[width=.8\maxwidth]{#1}}
    % Ensure that by default, figures have no caption (until we provide a
    % proper Figure object with a Caption API and a way to capture that
    % in the conversion process - todo).
    \usepackage{caption}
    \DeclareCaptionLabelFormat{nolabel}{}
    \captionsetup{labelformat=nolabel}

    \usepackage{adjustbox} % Used to constrain images to a maximum size 
    \usepackage{xcolor} % Allow colors to be defined
    \usepackage{enumerate} % Needed for markdown enumerations to work
    \usepackage{geometry} % Used to adjust the document margins
    \usepackage{amsmath} % Equations
    \usepackage{amssymb} % Equations
    \usepackage{textcomp} % defines textquotesingle
    % Hack from http://tex.stackexchange.com/a/47451/13684:
    \AtBeginDocument{%
        \def\PYZsq{\textquotesingle}% Upright quotes in Pygmentized code
    }
    \usepackage{upquote} % Upright quotes for verbatim code
    \usepackage{eurosym} % defines \euro
    \usepackage[mathletters]{ucs} % Extended unicode (utf-8) support
    \usepackage[utf8x]{inputenc} % Allow utf-8 characters in the tex document
    \usepackage{fancyvrb} % verbatim replacement that allows latex
    \usepackage{grffile} % extends the file name processing of package graphics 
                         % to support a larger range 
    % The hyperref package gives us a pdf with properly built
    % internal navigation ('pdf bookmarks' for the table of contents,
    % internal cross-reference links, web links for URLs, etc.)
    \usepackage{hyperref}
    \usepackage{longtable} % longtable support required by pandoc >1.10
    \usepackage{booktabs}  % table support for pandoc > 1.12.2
    \usepackage[inline]{enumitem} % IRkernel/repr support (it uses the enumerate* environment)
    \usepackage[normalem]{ulem} % ulem is needed to support strikethroughs (\sout)
                                % normalem makes italics be italics, not underlines
    

    
    
    % Colors for the hyperref package
    \definecolor{urlcolor}{rgb}{0,.145,.698}
    \definecolor{linkcolor}{rgb}{.71,0.21,0.01}
    \definecolor{citecolor}{rgb}{.12,.54,.11}

    % ANSI colors
    \definecolor{ansi-black}{HTML}{3E424D}
    \definecolor{ansi-black-intense}{HTML}{282C36}
    \definecolor{ansi-red}{HTML}{E75C58}
    \definecolor{ansi-red-intense}{HTML}{B22B31}
    \definecolor{ansi-green}{HTML}{00A250}
    \definecolor{ansi-green-intense}{HTML}{007427}
    \definecolor{ansi-yellow}{HTML}{DDB62B}
    \definecolor{ansi-yellow-intense}{HTML}{B27D12}
    \definecolor{ansi-blue}{HTML}{208FFB}
    \definecolor{ansi-blue-intense}{HTML}{0065CA}
    \definecolor{ansi-magenta}{HTML}{D160C4}
    \definecolor{ansi-magenta-intense}{HTML}{A03196}
    \definecolor{ansi-cyan}{HTML}{60C6C8}
    \definecolor{ansi-cyan-intense}{HTML}{258F8F}
    \definecolor{ansi-white}{HTML}{C5C1B4}
    \definecolor{ansi-white-intense}{HTML}{A1A6B2}

    % commands and environments needed by pandoc snippets
    % extracted from the output of `pandoc -s`
    \providecommand{\tightlist}{%
      \setlength{\itemsep}{0pt}\setlength{\parskip}{0pt}}
    \DefineVerbatimEnvironment{Highlighting}{Verbatim}{commandchars=\\\{\}}
    % Add ',fontsize=\small' for more characters per line
    \newenvironment{Shaded}{}{}
    \newcommand{\KeywordTok}[1]{\textcolor[rgb]{0.00,0.44,0.13}{\textbf{{#1}}}}
    \newcommand{\DataTypeTok}[1]{\textcolor[rgb]{0.56,0.13,0.00}{{#1}}}
    \newcommand{\DecValTok}[1]{\textcolor[rgb]{0.25,0.63,0.44}{{#1}}}
    \newcommand{\BaseNTok}[1]{\textcolor[rgb]{0.25,0.63,0.44}{{#1}}}
    \newcommand{\FloatTok}[1]{\textcolor[rgb]{0.25,0.63,0.44}{{#1}}}
    \newcommand{\CharTok}[1]{\textcolor[rgb]{0.25,0.44,0.63}{{#1}}}
    \newcommand{\StringTok}[1]{\textcolor[rgb]{0.25,0.44,0.63}{{#1}}}
    \newcommand{\CommentTok}[1]{\textcolor[rgb]{0.38,0.63,0.69}{\textit{{#1}}}}
    \newcommand{\OtherTok}[1]{\textcolor[rgb]{0.00,0.44,0.13}{{#1}}}
    \newcommand{\AlertTok}[1]{\textcolor[rgb]{1.00,0.00,0.00}{\textbf{{#1}}}}
    \newcommand{\FunctionTok}[1]{\textcolor[rgb]{0.02,0.16,0.49}{{#1}}}
    \newcommand{\RegionMarkerTok}[1]{{#1}}
    \newcommand{\ErrorTok}[1]{\textcolor[rgb]{1.00,0.00,0.00}{\textbf{{#1}}}}
    \newcommand{\NormalTok}[1]{{#1}}
    
    % Additional commands for more recent versions of Pandoc
    \newcommand{\ConstantTok}[1]{\textcolor[rgb]{0.53,0.00,0.00}{{#1}}}
    \newcommand{\SpecialCharTok}[1]{\textcolor[rgb]{0.25,0.44,0.63}{{#1}}}
    \newcommand{\VerbatimStringTok}[1]{\textcolor[rgb]{0.25,0.44,0.63}{{#1}}}
    \newcommand{\SpecialStringTok}[1]{\textcolor[rgb]{0.73,0.40,0.53}{{#1}}}
    \newcommand{\ImportTok}[1]{{#1}}
    \newcommand{\DocumentationTok}[1]{\textcolor[rgb]{0.73,0.13,0.13}{\textit{{#1}}}}
    \newcommand{\AnnotationTok}[1]{\textcolor[rgb]{0.38,0.63,0.69}{\textbf{\textit{{#1}}}}}
    \newcommand{\CommentVarTok}[1]{\textcolor[rgb]{0.38,0.63,0.69}{\textbf{\textit{{#1}}}}}
    \newcommand{\VariableTok}[1]{\textcolor[rgb]{0.10,0.09,0.49}{{#1}}}
    \newcommand{\ControlFlowTok}[1]{\textcolor[rgb]{0.00,0.44,0.13}{\textbf{{#1}}}}
    \newcommand{\OperatorTok}[1]{\textcolor[rgb]{0.40,0.40,0.40}{{#1}}}
    \newcommand{\BuiltInTok}[1]{{#1}}
    \newcommand{\ExtensionTok}[1]{{#1}}
    \newcommand{\PreprocessorTok}[1]{\textcolor[rgb]{0.74,0.48,0.00}{{#1}}}
    \newcommand{\AttributeTok}[1]{\textcolor[rgb]{0.49,0.56,0.16}{{#1}}}
    \newcommand{\InformationTok}[1]{\textcolor[rgb]{0.38,0.63,0.69}{\textbf{\textit{{#1}}}}}
    \newcommand{\WarningTok}[1]{\textcolor[rgb]{0.38,0.63,0.69}{\textbf{\textit{{#1}}}}}
    
    
    % Define a nice break command that doesn't care if a line doesn't already
    % exist.
    \def\br{\hspace*{\fill} \\* }
    % Math Jax compatability definitions
    \def\gt{>}
    \def\lt{<}
    % Document parameters
    \title{lab05\_audio\_partial}
    
    
    

    % Pygments definitions
    
\makeatletter
\def\PY@reset{\let\PY@it=\relax \let\PY@bf=\relax%
    \let\PY@ul=\relax \let\PY@tc=\relax%
    \let\PY@bc=\relax \let\PY@ff=\relax}
\def\PY@tok#1{\csname PY@tok@#1\endcsname}
\def\PY@toks#1+{\ifx\relax#1\empty\else%
    \PY@tok{#1}\expandafter\PY@toks\fi}
\def\PY@do#1{\PY@bc{\PY@tc{\PY@ul{%
    \PY@it{\PY@bf{\PY@ff{#1}}}}}}}
\def\PY#1#2{\PY@reset\PY@toks#1+\relax+\PY@do{#2}}

\expandafter\def\csname PY@tok@go\endcsname{\def\PY@tc##1{\textcolor[rgb]{0.53,0.53,0.53}{##1}}}
\expandafter\def\csname PY@tok@c\endcsname{\let\PY@it=\textit\def\PY@tc##1{\textcolor[rgb]{0.25,0.50,0.50}{##1}}}
\expandafter\def\csname PY@tok@mi\endcsname{\def\PY@tc##1{\textcolor[rgb]{0.40,0.40,0.40}{##1}}}
\expandafter\def\csname PY@tok@nv\endcsname{\def\PY@tc##1{\textcolor[rgb]{0.10,0.09,0.49}{##1}}}
\expandafter\def\csname PY@tok@kp\endcsname{\def\PY@tc##1{\textcolor[rgb]{0.00,0.50,0.00}{##1}}}
\expandafter\def\csname PY@tok@ch\endcsname{\let\PY@it=\textit\def\PY@tc##1{\textcolor[rgb]{0.25,0.50,0.50}{##1}}}
\expandafter\def\csname PY@tok@gu\endcsname{\let\PY@bf=\textbf\def\PY@tc##1{\textcolor[rgb]{0.50,0.00,0.50}{##1}}}
\expandafter\def\csname PY@tok@sr\endcsname{\def\PY@tc##1{\textcolor[rgb]{0.73,0.40,0.53}{##1}}}
\expandafter\def\csname PY@tok@gp\endcsname{\let\PY@bf=\textbf\def\PY@tc##1{\textcolor[rgb]{0.00,0.00,0.50}{##1}}}
\expandafter\def\csname PY@tok@sb\endcsname{\def\PY@tc##1{\textcolor[rgb]{0.73,0.13,0.13}{##1}}}
\expandafter\def\csname PY@tok@c1\endcsname{\let\PY@it=\textit\def\PY@tc##1{\textcolor[rgb]{0.25,0.50,0.50}{##1}}}
\expandafter\def\csname PY@tok@fm\endcsname{\def\PY@tc##1{\textcolor[rgb]{0.00,0.00,1.00}{##1}}}
\expandafter\def\csname PY@tok@cpf\endcsname{\let\PY@it=\textit\def\PY@tc##1{\textcolor[rgb]{0.25,0.50,0.50}{##1}}}
\expandafter\def\csname PY@tok@gs\endcsname{\let\PY@bf=\textbf}
\expandafter\def\csname PY@tok@cp\endcsname{\def\PY@tc##1{\textcolor[rgb]{0.74,0.48,0.00}{##1}}}
\expandafter\def\csname PY@tok@vm\endcsname{\def\PY@tc##1{\textcolor[rgb]{0.10,0.09,0.49}{##1}}}
\expandafter\def\csname PY@tok@sx\endcsname{\def\PY@tc##1{\textcolor[rgb]{0.00,0.50,0.00}{##1}}}
\expandafter\def\csname PY@tok@s\endcsname{\def\PY@tc##1{\textcolor[rgb]{0.73,0.13,0.13}{##1}}}
\expandafter\def\csname PY@tok@nd\endcsname{\def\PY@tc##1{\textcolor[rgb]{0.67,0.13,1.00}{##1}}}
\expandafter\def\csname PY@tok@cm\endcsname{\let\PY@it=\textit\def\PY@tc##1{\textcolor[rgb]{0.25,0.50,0.50}{##1}}}
\expandafter\def\csname PY@tok@cs\endcsname{\let\PY@it=\textit\def\PY@tc##1{\textcolor[rgb]{0.25,0.50,0.50}{##1}}}
\expandafter\def\csname PY@tok@nn\endcsname{\let\PY@bf=\textbf\def\PY@tc##1{\textcolor[rgb]{0.00,0.00,1.00}{##1}}}
\expandafter\def\csname PY@tok@err\endcsname{\def\PY@bc##1{\setlength{\fboxsep}{0pt}\fcolorbox[rgb]{1.00,0.00,0.00}{1,1,1}{\strut ##1}}}
\expandafter\def\csname PY@tok@o\endcsname{\def\PY@tc##1{\textcolor[rgb]{0.40,0.40,0.40}{##1}}}
\expandafter\def\csname PY@tok@sc\endcsname{\def\PY@tc##1{\textcolor[rgb]{0.73,0.13,0.13}{##1}}}
\expandafter\def\csname PY@tok@sd\endcsname{\let\PY@it=\textit\def\PY@tc##1{\textcolor[rgb]{0.73,0.13,0.13}{##1}}}
\expandafter\def\csname PY@tok@vi\endcsname{\def\PY@tc##1{\textcolor[rgb]{0.10,0.09,0.49}{##1}}}
\expandafter\def\csname PY@tok@ne\endcsname{\let\PY@bf=\textbf\def\PY@tc##1{\textcolor[rgb]{0.82,0.25,0.23}{##1}}}
\expandafter\def\csname PY@tok@nl\endcsname{\def\PY@tc##1{\textcolor[rgb]{0.63,0.63,0.00}{##1}}}
\expandafter\def\csname PY@tok@nb\endcsname{\def\PY@tc##1{\textcolor[rgb]{0.00,0.50,0.00}{##1}}}
\expandafter\def\csname PY@tok@sh\endcsname{\def\PY@tc##1{\textcolor[rgb]{0.73,0.13,0.13}{##1}}}
\expandafter\def\csname PY@tok@il\endcsname{\def\PY@tc##1{\textcolor[rgb]{0.40,0.40,0.40}{##1}}}
\expandafter\def\csname PY@tok@gr\endcsname{\def\PY@tc##1{\textcolor[rgb]{1.00,0.00,0.00}{##1}}}
\expandafter\def\csname PY@tok@nc\endcsname{\let\PY@bf=\textbf\def\PY@tc##1{\textcolor[rgb]{0.00,0.00,1.00}{##1}}}
\expandafter\def\csname PY@tok@kt\endcsname{\def\PY@tc##1{\textcolor[rgb]{0.69,0.00,0.25}{##1}}}
\expandafter\def\csname PY@tok@ge\endcsname{\let\PY@it=\textit}
\expandafter\def\csname PY@tok@vg\endcsname{\def\PY@tc##1{\textcolor[rgb]{0.10,0.09,0.49}{##1}}}
\expandafter\def\csname PY@tok@no\endcsname{\def\PY@tc##1{\textcolor[rgb]{0.53,0.00,0.00}{##1}}}
\expandafter\def\csname PY@tok@gt\endcsname{\def\PY@tc##1{\textcolor[rgb]{0.00,0.27,0.87}{##1}}}
\expandafter\def\csname PY@tok@mb\endcsname{\def\PY@tc##1{\textcolor[rgb]{0.40,0.40,0.40}{##1}}}
\expandafter\def\csname PY@tok@vc\endcsname{\def\PY@tc##1{\textcolor[rgb]{0.10,0.09,0.49}{##1}}}
\expandafter\def\csname PY@tok@k\endcsname{\let\PY@bf=\textbf\def\PY@tc##1{\textcolor[rgb]{0.00,0.50,0.00}{##1}}}
\expandafter\def\csname PY@tok@m\endcsname{\def\PY@tc##1{\textcolor[rgb]{0.40,0.40,0.40}{##1}}}
\expandafter\def\csname PY@tok@ni\endcsname{\let\PY@bf=\textbf\def\PY@tc##1{\textcolor[rgb]{0.60,0.60,0.60}{##1}}}
\expandafter\def\csname PY@tok@gi\endcsname{\def\PY@tc##1{\textcolor[rgb]{0.00,0.63,0.00}{##1}}}
\expandafter\def\csname PY@tok@sa\endcsname{\def\PY@tc##1{\textcolor[rgb]{0.73,0.13,0.13}{##1}}}
\expandafter\def\csname PY@tok@kr\endcsname{\let\PY@bf=\textbf\def\PY@tc##1{\textcolor[rgb]{0.00,0.50,0.00}{##1}}}
\expandafter\def\csname PY@tok@se\endcsname{\let\PY@bf=\textbf\def\PY@tc##1{\textcolor[rgb]{0.73,0.40,0.13}{##1}}}
\expandafter\def\csname PY@tok@bp\endcsname{\def\PY@tc##1{\textcolor[rgb]{0.00,0.50,0.00}{##1}}}
\expandafter\def\csname PY@tok@mh\endcsname{\def\PY@tc##1{\textcolor[rgb]{0.40,0.40,0.40}{##1}}}
\expandafter\def\csname PY@tok@mo\endcsname{\def\PY@tc##1{\textcolor[rgb]{0.40,0.40,0.40}{##1}}}
\expandafter\def\csname PY@tok@nf\endcsname{\def\PY@tc##1{\textcolor[rgb]{0.00,0.00,1.00}{##1}}}
\expandafter\def\csname PY@tok@si\endcsname{\let\PY@bf=\textbf\def\PY@tc##1{\textcolor[rgb]{0.73,0.40,0.53}{##1}}}
\expandafter\def\csname PY@tok@kc\endcsname{\let\PY@bf=\textbf\def\PY@tc##1{\textcolor[rgb]{0.00,0.50,0.00}{##1}}}
\expandafter\def\csname PY@tok@gd\endcsname{\def\PY@tc##1{\textcolor[rgb]{0.63,0.00,0.00}{##1}}}
\expandafter\def\csname PY@tok@mf\endcsname{\def\PY@tc##1{\textcolor[rgb]{0.40,0.40,0.40}{##1}}}
\expandafter\def\csname PY@tok@na\endcsname{\def\PY@tc##1{\textcolor[rgb]{0.49,0.56,0.16}{##1}}}
\expandafter\def\csname PY@tok@ow\endcsname{\let\PY@bf=\textbf\def\PY@tc##1{\textcolor[rgb]{0.67,0.13,1.00}{##1}}}
\expandafter\def\csname PY@tok@gh\endcsname{\let\PY@bf=\textbf\def\PY@tc##1{\textcolor[rgb]{0.00,0.00,0.50}{##1}}}
\expandafter\def\csname PY@tok@w\endcsname{\def\PY@tc##1{\textcolor[rgb]{0.73,0.73,0.73}{##1}}}
\expandafter\def\csname PY@tok@s1\endcsname{\def\PY@tc##1{\textcolor[rgb]{0.73,0.13,0.13}{##1}}}
\expandafter\def\csname PY@tok@nt\endcsname{\let\PY@bf=\textbf\def\PY@tc##1{\textcolor[rgb]{0.00,0.50,0.00}{##1}}}
\expandafter\def\csname PY@tok@s2\endcsname{\def\PY@tc##1{\textcolor[rgb]{0.73,0.13,0.13}{##1}}}
\expandafter\def\csname PY@tok@kd\endcsname{\let\PY@bf=\textbf\def\PY@tc##1{\textcolor[rgb]{0.00,0.50,0.00}{##1}}}
\expandafter\def\csname PY@tok@ss\endcsname{\def\PY@tc##1{\textcolor[rgb]{0.10,0.09,0.49}{##1}}}
\expandafter\def\csname PY@tok@kn\endcsname{\let\PY@bf=\textbf\def\PY@tc##1{\textcolor[rgb]{0.00,0.50,0.00}{##1}}}
\expandafter\def\csname PY@tok@dl\endcsname{\def\PY@tc##1{\textcolor[rgb]{0.73,0.13,0.13}{##1}}}

\def\PYZbs{\char`\\}
\def\PYZus{\char`\_}
\def\PYZob{\char`\{}
\def\PYZcb{\char`\}}
\def\PYZca{\char`\^}
\def\PYZam{\char`\&}
\def\PYZlt{\char`\<}
\def\PYZgt{\char`\>}
\def\PYZsh{\char`\#}
\def\PYZpc{\char`\%}
\def\PYZdl{\char`\$}
\def\PYZhy{\char`\-}
\def\PYZsq{\char`\'}
\def\PYZdq{\char`\"}
\def\PYZti{\char`\~}
% for compatibility with earlier versions
\def\PYZat{@}
\def\PYZlb{[}
\def\PYZrb{]}
\makeatother


    % Exact colors from NB
    \definecolor{incolor}{rgb}{0.0, 0.0, 0.5}
    \definecolor{outcolor}{rgb}{0.545, 0.0, 0.0}



    
    % Prevent overflowing lines due to hard-to-break entities
    \sloppy 
    % Setup hyperref package
    \hypersetup{
      breaklinks=true,  % so long urls are correctly broken across lines
      colorlinks=true,
      urlcolor=urlcolor,
      linkcolor=linkcolor,
      citecolor=citecolor,
      }
    % Slightly bigger margins than the latex defaults
    
    \geometry{verbose,tmargin=1in,bmargin=1in,lmargin=1in,rmargin=1in}
    
    

    \begin{document}
    
    
    \maketitle
    
    

    
    \section{Lab 5: Pitch Detection in
Audio}\label{lab-5-pitch-detection-in-audio}

In this lab, we will use numerical optimization to find the pitch and
harmonics in a simple audio signal. In addition to the concepts in the
\href{./grad_descent.ipynb}{gradient descent demo}, you will learn to: *
Load, visualize and play audio recordings * Divide audio data into
frames * Perform nested minimization

The ML method presented here for pitch detection is actually not a very
good one. As we will see, it is highly susceptible to local minima and
quite slow. There are several better
\href{https://en.wikipedia.org/wiki/Pitch_detection_algorithm}{pitch
detection algorithms}, mostly using frequency-domain techniques. But,
the method here will illustrate non-linear estimation well.

    \subsection{Reading the Audio File}\label{reading-the-audio-file}

Python provides a very simple method to read a \texttt{wav} file in the
\texttt{scipy.io.wavefile} package. We first load that along with the
other packages.

    \begin{Verbatim}[commandchars=\\\{\}]
{\color{incolor}In [{\color{incolor}1}]:} \PY{k+kn}{from} \PY{n+nn}{scipy}\PY{n+nn}{.}\PY{n+nn}{io}\PY{n+nn}{.}\PY{n+nn}{wavfile} \PY{k}{import} \PY{n}{read}
        \PY{k+kn}{import} \PY{n+nn}{numpy} \PY{k}{as} \PY{n+nn}{np}
        \PY{k+kn}{import} \PY{n+nn}{matplotlib}\PY{n+nn}{.}\PY{n+nn}{pyplot} \PY{k}{as} \PY{n+nn}{plt}
        \PY{o}{\PYZpc{}}\PY{k}{matplotlib} inline
\end{Verbatim}


    In the github repository, you should find a file,
\href{./viola.wav}{\texttt{viola.wav}}. Download this file to your local
directory. Although the file is included in the github repository, you
can find it along with many other audio samples in
\href{https://ccrma.stanford.edu/~jos/pasp/Sound_Examples.html}{CCRMA
audio website}. After you have downloaded the file, you can then read
the file with the \texttt{read} command. Print the sample rate in Hz,
the number of samples in the file and the file length in seconds.

    \begin{Verbatim}[commandchars=\\\{\}]
{\color{incolor}In [{\color{incolor}2}]:} \PY{c+c1}{\PYZsh{} Read the file}
        \PY{n}{sr}\PY{p}{,} \PY{n}{y} \PY{o}{=} \PY{n}{read}\PY{p}{(}\PY{l+s+s1}{\PYZsq{}}\PY{l+s+s1}{viola.wav}\PY{l+s+s1}{\PYZsq{}}\PY{p}{)}
        
        \PY{c+c1}{\PYZsh{} Convert to floating point values so that compuations below do not overflow}
        \PY{n}{y} \PY{o}{=} \PY{n}{y}\PY{o}{.}\PY{n}{astype}\PY{p}{(}\PY{n+nb}{float}\PY{p}{)}
        
        \PY{c+c1}{\PYZsh{} TODO:  Print sample rate, number of samples and file length in seconds.}
        \PY{n+nb}{print}\PY{p}{(}\PY{l+s+s1}{\PYZsq{}}\PY{l+s+s1}{Sample rate: }\PY{l+s+si}{\PYZpc{}d}\PY{l+s+s1}{Hz}\PY{l+s+s1}{\PYZsq{}} \PY{o}{\PYZpc{}} \PY{n}{sr}\PY{p}{)}
        \PY{n+nb}{print}\PY{p}{(}\PY{l+s+s1}{\PYZsq{}}\PY{l+s+s1}{The number of samples: }\PY{l+s+si}{\PYZpc{}d}\PY{l+s+s1}{\PYZsq{}} \PY{o}{\PYZpc{}} \PY{n+nb}{len}\PY{p}{(}\PY{n}{y}\PY{p}{)}\PY{p}{)}
        \PY{n+nb}{print}\PY{p}{(}\PY{l+s+s1}{\PYZsq{}}\PY{l+s+s1}{The file length: }\PY{l+s+si}{\PYZpc{}.4f}\PY{l+s+s1}{s}\PY{l+s+s1}{\PYZsq{}} \PY{o}{\PYZpc{}} \PY{p}{(}\PY{n+nb}{len}\PY{p}{(}\PY{n}{y}\PY{p}{)}\PY{o}{/}\PY{n}{sr}\PY{p}{)}\PY{p}{)}
\end{Verbatim}


    \begin{Verbatim}[commandchars=\\\{\}]
Sample rate: 44100Hz
The number of samples: 299350
The file length: 6.7880s

    \end{Verbatim}

    You can then play the file with the following command. You should hear
the viola play a sequence of simple notes.

    \begin{Verbatim}[commandchars=\\\{\}]
{\color{incolor}In [{\color{incolor}3}]:} \PY{k+kn}{import} \PY{n+nn}{IPython}\PY{n+nn}{.}\PY{n+nn}{display} \PY{k}{as} \PY{n+nn}{ipd}
        \PY{n}{ipd}\PY{o}{.}\PY{n}{Audio}\PY{p}{(}\PY{n}{y}\PY{p}{,} \PY{n}{rate}\PY{o}{=}\PY{n}{sr}\PY{p}{)} \PY{c+c1}{\PYZsh{} load a NumPy array}
\end{Verbatim}


\begin{Verbatim}[commandchars=\\\{\}]
{\color{outcolor}Out[{\color{outcolor}3}]:} <IPython.lib.display.Audio object>
\end{Verbatim}
            
    For the analysis below, it will be easier to re-scale the samples so
that they have an average squared value of 1. Find the \texttt{scale}
value in the code below to do this.

    \begin{Verbatim}[commandchars=\\\{\}]
{\color{incolor}In [{\color{incolor}4}]:} \PY{c+c1}{\PYZsh{} TODO}
        \PY{n}{scale} \PY{o}{=} \PY{n}{np}\PY{o}{.}\PY{n}{sqrt}\PY{p}{(}\PY{n}{np}\PY{o}{.}\PY{n}{sum}\PY{p}{(}\PY{p}{(}\PY{n}{y}\PY{o}{/}\PY{n}{np}\PY{o}{.}\PY{n}{sqrt}\PY{p}{(}\PY{n+nb}{len}\PY{p}{(}\PY{n}{y}\PY{p}{)}\PY{p}{)}\PY{p}{)}\PY{o}{*}\PY{o}{*}\PY{l+m+mi}{2}\PY{p}{)}\PY{p}{)}
        \PY{n}{y} \PY{o}{=} \PY{n}{y} \PY{o}{/} \PY{n}{scale}
\end{Verbatim}


    \subsection{Dividing the Audio File into
Frames}\label{dividing-the-audio-file-into-frames}

In audio processing, it is common to divide audio streams into short
frames (typically between 10 to 40 ms long). Since frames are often
processed with an FFT, the frames are typically a power of two. Analysis
is then performed in the frames separately. Given the vector \texttt{y},
create a \texttt{nfft\ x\ nframe} matrix \texttt{yframe} where

\begin{verbatim}
yframe[:,0] = samples y[k], k=0,...,nfft-1
yframe[:,1] = samples y[k], k=nfft,...,2*nfft-1,
yframe[:,2] = samples y[k], k=2*nfft,...,3*nfft-1,
...
\end{verbatim}

You can do this with the \texttt{reshape} command with \texttt{order=F}.
Zero pad \texttt{y} if the number of samples of \texttt{y} is not
divisible by \texttt{nfft}. Print the total number of frames as well as
the length (in milliseconds) of each frame.

Note that in actual audio processing, the frames are typically
overlapping and use careful windowing. But, we will ignore that here for
simplicity.

    \begin{Verbatim}[commandchars=\\\{\}]
{\color{incolor}In [{\color{incolor}5}]:} \PY{c+c1}{\PYZsh{} Frame size}
        \PY{n}{nfft} \PY{o}{=} \PY{l+m+mi}{1024}
        
        \PY{c+c1}{\PYZsh{} TODO:}
        \PY{k+kn}{import} \PY{n+nn}{math}
        \PY{n}{nframe} \PY{o}{=} \PY{n}{math}\PY{o}{.}\PY{n}{ceil}\PY{p}{(}\PY{n+nb}{len}\PY{p}{(}\PY{n}{y}\PY{p}{)} \PY{o}{/} \PY{n}{nfft}\PY{p}{)}
        \PY{n}{yframe} \PY{o}{=} \PY{n}{np}\PY{o}{.}\PY{n}{pad}\PY{p}{(}\PY{n}{y}\PY{p}{,} \PY{p}{(}\PY{l+m+mi}{0}\PY{p}{,} \PY{n}{nfft} \PY{o}{\PYZhy{}} \PY{n+nb}{len}\PY{p}{(}\PY{n}{y}\PY{p}{)} \PY{o}{\PYZpc{}} \PY{n}{nfft}\PY{p}{)}\PY{p}{,} \PY{l+s+s1}{\PYZsq{}}\PY{l+s+s1}{constant}\PY{l+s+s1}{\PYZsq{}}\PY{p}{)}\PY{o}{.}\PY{n}{reshape}\PY{p}{(}\PY{n}{nfft}\PY{p}{,} \PY{n}{nframe}\PY{p}{,} \PY{n}{order}\PY{o}{=}\PY{l+s+s1}{\PYZsq{}}\PY{l+s+s1}{F}\PY{l+s+s1}{\PYZsq{}}\PY{p}{)}
        \PY{n+nb}{print}\PY{p}{(}\PY{l+s+s1}{\PYZsq{}}\PY{l+s+s1}{Total number of frames: }\PY{l+s+si}{\PYZpc{}d}\PY{l+s+s1}{\PYZsq{}} \PY{o}{\PYZpc{}} \PY{n}{nframe}\PY{p}{)}
        \PY{n+nb}{print}\PY{p}{(}\PY{l+s+s1}{\PYZsq{}}\PY{l+s+s1}{Length of each frame: }\PY{l+s+si}{\PYZpc{}.4f}\PY{l+s+s1}{ms}\PY{l+s+s1}{\PYZsq{}} \PY{o}{\PYZpc{}} \PY{p}{(}\PY{n}{nfft}\PY{o}{/}\PY{n}{sr}\PY{o}{*}\PY{l+m+mi}{1000}\PY{p}{)}\PY{p}{)}
\end{Verbatim}


    \begin{Verbatim}[commandchars=\\\{\}]
Total number of frames: 293
Length of each frame: 23.2200ms

    \end{Verbatim}

    Let \texttt{i0=10} and set \texttt{yi=yframe{[}:,i0{]}} be the samples
of frame \texttt{i0}. We will use this frame for most of the rest of the
lab. Plot the samples of \texttt{yi}. Label the time axis in
milliseconds (ms).

    \begin{Verbatim}[commandchars=\\\{\}]
{\color{incolor}In [{\color{incolor}6}]:} \PY{c+c1}{\PYZsh{} Get samples from frame 10}
        \PY{n}{i0} \PY{o}{=} \PY{l+m+mi}{10}
        \PY{n}{yi} \PY{o}{=} \PY{n}{yframe}\PY{p}{[}\PY{p}{:}\PY{p}{,}\PY{n}{i0}\PY{p}{]}
        
        \PY{c+c1}{\PYZsh{} TODO:  Plot yi vs. time (in ms)}
        \PY{n}{plt}\PY{o}{.}\PY{n}{plot}\PY{p}{(}\PY{n}{np}\PY{o}{.}\PY{n}{array}\PY{p}{(}\PY{n+nb}{range}\PY{p}{(}\PY{n+nb}{len}\PY{p}{(}\PY{n}{yi}\PY{p}{)}\PY{p}{)}\PY{p}{)}\PY{o}{/}\PY{n}{sr}\PY{o}{*}\PY{l+m+mi}{1000}\PY{p}{,} \PY{n}{yi}\PY{p}{)}
        \PY{n}{plt}\PY{o}{.}\PY{n}{xlabel}\PY{p}{(}\PY{l+s+s1}{\PYZsq{}}\PY{l+s+s1}{Time(ms)}\PY{l+s+s1}{\PYZsq{}}\PY{p}{)}
        \PY{n}{plt}\PY{o}{.}\PY{n}{ylabel}\PY{p}{(}\PY{l+s+s1}{\PYZsq{}}\PY{l+s+s1}{Amplitude}\PY{l+s+s1}{\PYZsq{}}\PY{p}{)}
        \PY{n}{plt}\PY{o}{.}\PY{n}{show}\PY{p}{(}\PY{p}{)}
\end{Verbatim}


    \begin{center}
    \adjustimage{max size={0.9\linewidth}{0.9\paperheight}}{output_12_0.png}
    \end{center}
    { \hspace*{\fill} \\}
    
    \subsection{Fitting a Multi-Sinusoid}\label{fitting-a-multi-sinusoid}

A common model for audio samples, \texttt{yi{[}k{]}}, containing an
instrument playing a single note is the multi-sinusoid model:

\begin{verbatim}
yi[k] \approx yhati[k] = c + \sum_{j=0}^{nterms-1} a[j]*cos(2*np.pi*k*freq0*(j+1)/sr) 
                                                +  b[j]*sin(2*np.pi*k*freq0*(j+1)/sr),
    
\end{verbatim}

where \texttt{sr} is the sample rate. The parameter \texttt{freq0} is
called the fundamental frequency and the audio signal is modeled as
being composed of sinusoids and cosinusoids with frequencies equal to
integer multiples of the fundamental. In audio processing, these terms
are called \emph{harmonics}. In analyzing audio signals, a common goal
is to determine both the fundamental frequency \texttt{freq0} (the pitch
of the audio) as well as the coefficients of the harmonics,

\begin{verbatim}
beta = (c, a[0], ..., a[nterms-1], b[0], ..., b[nterms-1]).
\end{verbatim}

To find the parameters, we will fit the mean squared error loss
function:

\begin{verbatim}
mse(freq0,beta) := 1/N * \sum_k (yi[k] - yhati[k])**2,   N = len(yi).
\end{verbatim}

In practice, a separate model would be fit for each audio frame. But, in
this lab, we will mostly look at a single frame.

\subsubsection{Nested Minimization}\label{nested-minimization}

We will perform the minimization of \texttt{mse} in a nested manner:
First, given a fundamental frequency \texttt{freq0}, we minimize over
the coefficients \texttt{beta}. Call this minimum \texttt{mse1}:

\begin{verbatim}
mse1(freq0) := min_beta mse(freq0,beta)
\end{verbatim}

Importantly, this minimizaiton can be performed by least-squares. Then,
we find the fundamental frequency \texttt{freq0} by minimizing
\texttt{mse1}:

\begin{verbatim}
min_{freq0} mse1(freq0) 
\end{verbatim}

We will use gradient-descent minimization with \texttt{mse1(freq0)} as
the objective function. This form of \emph{nested} minimization is
commonly used whenever we can minimize over one set of parameters easily
given the other.

    \subsection{Setting Up the Objective
Function}\label{setting-up-the-objective-function}

We will use the class \texttt{AudioFitFn} below to perform the two-part
minimization. Complete the \texttt{feval} method in the class. The
method should take the argument \texttt{freq0} and perform the
minimization of the MSE over \texttt{beta}. Specifically, fill the code
in \texttt{feval} to perform the following: * Construct a matrix,
\texttt{A} such that \texttt{yhati\ =\ A*beta}.\\
* Find \texttt{betahat} with the \texttt{np.linalg.lstsq()} method using
the matrix \texttt{A} and the samples \texttt{self.yi}. This is simpler
than constructing a linear regression object.\\
* Compute and store the estimate \texttt{self.yhati\ =\ A.dot(betahat)}.
* Compute the \texttt{mse1}, the minimum MSE, by comparing
\texttt{self.yhati} and \texttt{self.yi}. * For now, set the gradient to
\texttt{mse1\_grad=0}. We will fill this part in later.\\
* Return \texttt{mse1} and \texttt{mse1\_grad}.

    \begin{Verbatim}[commandchars=\\\{\}]
{\color{incolor}In [{\color{incolor}7}]:} \PY{k}{class} \PY{n+nc}{AudioFitFn}\PY{p}{(}\PY{n+nb}{object}\PY{p}{)}\PY{p}{:}
            \PY{k}{def} \PY{n+nf}{\PYZus{}\PYZus{}init\PYZus{}\PYZus{}}\PY{p}{(}\PY{n+nb+bp}{self}\PY{p}{,}\PY{n}{yi}\PY{p}{,}\PY{n}{sr}\PY{o}{=}\PY{l+m+mi}{44100}\PY{p}{,}\PY{n}{nterms}\PY{o}{=}\PY{l+m+mi}{8}\PY{p}{)}\PY{p}{:}
                \PY{l+s+sd}{\PYZdq{}\PYZdq{}\PYZdq{}}
        \PY{l+s+sd}{        A class for fitting }
        \PY{l+s+sd}{        }
        \PY{l+s+sd}{        yi:  One frame of audio}
        \PY{l+s+sd}{        sr:  Sample rate (in Hz)}
        \PY{l+s+sd}{        nterms:  Number of harmonics used in the model (default=8)}
        \PY{l+s+sd}{        \PYZdq{}\PYZdq{}\PYZdq{}}
                \PY{n+nb+bp}{self}\PY{o}{.}\PY{n}{yi} \PY{o}{=} \PY{n}{yi}
                \PY{n+nb+bp}{self}\PY{o}{.}\PY{n}{sr} \PY{o}{=} \PY{n}{sr}
                \PY{n+nb+bp}{self}\PY{o}{.}\PY{n}{nterms} \PY{o}{=} \PY{n}{nterms}
                        
            \PY{k}{def} \PY{n+nf}{feval}\PY{p}{(}\PY{n+nb+bp}{self}\PY{p}{,}\PY{n}{freq0}\PY{p}{)}\PY{p}{:}
                \PY{l+s+sd}{\PYZdq{}\PYZdq{}\PYZdq{}}
        \PY{l+s+sd}{        Optimization function for audio fitting.  Given a fundamental frequency, freq0, the }
        \PY{l+s+sd}{        method performs a least squares fit for the audio sample using the model:}
        \PY{l+s+sd}{        }
        \PY{l+s+sd}{        yhati[k] = c + \PYZbs{}sum\PYZus{}\PYZob{}j=0\PYZcb{}\PYZca{}\PYZob{}nterms\PYZhy{}1\PYZcb{} a[j]*cos(2*np.pi*k*freq0*(j+1)/sr) }
        \PY{l+s+sd}{                                          +  b[j]*sin(2*np.pi*k*freq0*(j+1)/sr)}
        \PY{l+s+sd}{        }
        \PY{l+s+sd}{        The coefficients beta = [c,a[0],...,a[nterms\PYZhy{}1],b[0],...,b[nterms\PYZhy{}1]] }
        \PY{l+s+sd}{        are found by least squares.}
        
        \PY{l+s+sd}{        Returns:}
        \PY{l+s+sd}{        }
        \PY{l+s+sd}{        mse1:   The MSE of the best least square fit.}
        \PY{l+s+sd}{        mse1\PYZus{}grad:  The gradient of mse1 wrt to the parameter freq0}
        \PY{l+s+sd}{        \PYZdq{}\PYZdq{}\PYZdq{}}
                
                \PY{c+c1}{\PYZsh{} TODO   }
                \PY{c+c1}{\PYZsh{} TODO Write code to find optimal beta to minimize MSE and find minimal MSE with this beta}
                \PY{n}{cos\PYZus{}vals} \PY{o}{=} \PY{p}{[}\PY{p}{[}\PY{n}{np}\PY{o}{.}\PY{n}{cos}\PY{p}{(}\PY{l+m+mi}{2}\PY{o}{*}\PY{n}{np}\PY{o}{.}\PY{n}{pi}\PY{o}{*}\PY{n}{k}\PY{o}{*}\PY{n}{freq0}\PY{o}{*}\PY{p}{(}\PY{n}{j}\PY{o}{+}\PY{l+m+mi}{1}\PY{p}{)}\PY{o}{/}\PY{n+nb+bp}{self}\PY{o}{.}\PY{n}{sr}\PY{p}{)} \PYZbs{}
                             \PY{k}{for} \PY{n}{j} \PY{o+ow}{in} \PY{n+nb}{range}\PY{p}{(}\PY{n+nb+bp}{self}\PY{o}{.}\PY{n}{nterms}\PY{p}{)}\PY{p}{]} \PY{k}{for} \PY{n}{k} \PY{o+ow}{in} \PY{n+nb}{range}\PY{p}{(}\PY{n+nb}{len}\PY{p}{(}\PY{n+nb+bp}{self}\PY{o}{.}\PY{n}{yi}\PY{p}{)}\PY{p}{)}\PY{p}{]}
                \PY{n}{sin\PYZus{}vals} \PY{o}{=} \PY{p}{[}\PY{p}{[}\PY{n}{np}\PY{o}{.}\PY{n}{sin}\PY{p}{(}\PY{l+m+mi}{2}\PY{o}{*}\PY{n}{np}\PY{o}{.}\PY{n}{pi}\PY{o}{*}\PY{n}{k}\PY{o}{*}\PY{n}{freq0}\PY{o}{*}\PY{p}{(}\PY{n}{j}\PY{o}{+}\PY{l+m+mi}{1}\PY{p}{)}\PY{o}{/}\PY{n+nb+bp}{self}\PY{o}{.}\PY{n}{sr}\PY{p}{)} \PYZbs{}
                             \PY{k}{for} \PY{n}{j} \PY{o+ow}{in} \PY{n+nb}{range}\PY{p}{(}\PY{n+nb+bp}{self}\PY{o}{.}\PY{n}{nterms}\PY{p}{)}\PY{p}{]} \PY{k}{for} \PY{n}{k} \PY{o+ow}{in} \PY{n+nb}{range}\PY{p}{(}\PY{n+nb}{len}\PY{p}{(}\PY{n+nb+bp}{self}\PY{o}{.}\PY{n}{yi}\PY{p}{)}\PY{p}{)}\PY{p}{]}
                \PY{n}{A} \PY{o}{=} \PY{n}{np}\PY{o}{.}\PY{n}{column\PYZus{}stack}\PY{p}{(}\PY{p}{(}\PY{n}{np}\PY{o}{.}\PY{n}{ones}\PY{p}{(}\PY{n+nb}{len}\PY{p}{(}\PY{n+nb+bp}{self}\PY{o}{.}\PY{n}{yi}\PY{p}{)}\PY{p}{,}\PY{p}{)}\PY{p}{,} \PY{n}{cos\PYZus{}vals}\PY{p}{,} \PY{n}{sin\PYZus{}vals}\PY{p}{)}\PY{p}{)}
                \PY{n}{beta} \PY{o}{=} \PY{n}{np}\PY{o}{.}\PY{n}{linalg}\PY{o}{.}\PY{n}{lstsq}\PY{p}{(}\PY{n}{A}\PY{p}{,} \PY{n+nb+bp}{self}\PY{o}{.}\PY{n}{yi}\PY{p}{,} \PY{n}{rcond}\PY{o}{=}\PY{k+kc}{None}\PY{p}{)}
                \PY{n+nb+bp}{self}\PY{o}{.}\PY{n}{yhati} \PY{o}{=} \PY{n}{A}\PY{o}{.}\PY{n}{dot}\PY{p}{(}\PY{n}{beta}\PY{p}{[}\PY{l+m+mi}{0}\PY{p}{]}\PY{p}{)}
                \PY{n}{mse1} \PY{o}{=} \PY{n}{np}\PY{o}{.}\PY{n}{mean}\PY{p}{(}\PY{p}{(}\PY{n+nb+bp}{self}\PY{o}{.}\PY{n}{yi}\PY{o}{\PYZhy{}}\PY{n+nb+bp}{self}\PY{o}{.}\PY{n}{yhati}\PY{p}{)}\PY{o}{*}\PY{o}{*}\PY{l+m+mi}{2}\PY{p}{)}
            
                \PY{c+c1}{\PYZsh{} Compute the gradient wrt to freq0 }
                \PY{n}{mse1\PYZus{}grad} \PY{o}{=} \PY{l+m+mi}{0}
                \PY{k}{return} \PY{n}{mse1}\PY{p}{,} \PY{n}{mse1\PYZus{}grad}
\end{Verbatim}


    Instantiate an object, \texttt{audio\_fn} from the class
\texttt{AudioFitFn} with the samples \texttt{yi}. Then, using the
\texttt{feval} method, compute and plot \texttt{mse1} for values
\texttt{freq0} in the range of 0 to 500 Hz with a step size of 0.5Hz.
You should see a minimum around \texttt{freq0\ =\ 131} Hz, but there are
several other local minima.

    \begin{Verbatim}[commandchars=\\\{\}]
{\color{incolor}In [{\color{incolor}8}]:} \PY{c+c1}{\PYZsh{} TODO}
        \PY{n}{audio\PYZus{}fn} \PY{o}{=} \PY{n}{AudioFitFn}\PY{p}{(}\PY{n}{yi}\PY{p}{,} \PY{n}{sr}\PY{p}{)}
        \PY{n}{freq0} \PY{o}{=} \PY{n}{np}\PY{o}{.}\PY{n}{linspace}\PY{p}{(}\PY{l+m+mi}{40}\PY{p}{,} \PY{l+m+mi}{500}\PY{p}{,} \PY{l+m+mi}{100}\PY{p}{)}
        \PY{n}{mse1} \PY{o}{=} \PY{n}{np}\PY{o}{.}\PY{n}{array}\PY{p}{(}\PY{p}{[}\PY{n}{audio\PYZus{}fn}\PY{o}{.}\PY{n}{feval}\PY{p}{(}\PY{n}{f}\PY{p}{)}\PY{p}{[}\PY{l+m+mi}{0}\PY{p}{]} \PY{k}{for} \PY{n}{f} \PY{o+ow}{in} \PY{n}{freq0}\PY{p}{]}\PY{p}{)}
        \PY{n}{plt}\PY{o}{.}\PY{n}{plot}\PY{p}{(}\PY{n}{freq0}\PY{p}{,} \PY{n}{mse1}\PY{p}{)}
        \PY{n}{plt}\PY{o}{.}\PY{n}{xlabel}\PY{p}{(}\PY{l+s+s1}{\PYZsq{}}\PY{l+s+s1}{freq0}\PY{l+s+s1}{\PYZsq{}}\PY{p}{)}
        \PY{n}{plt}\PY{o}{.}\PY{n}{ylabel}\PY{p}{(}\PY{l+s+s1}{\PYZsq{}}\PY{l+s+s1}{mse1}\PY{l+s+s1}{\PYZsq{}}\PY{p}{)}
        \PY{n}{plt}\PY{o}{.}\PY{n}{show}\PY{p}{(}\PY{p}{)}
\end{Verbatim}


    \begin{center}
    \adjustimage{max size={0.9\linewidth}{0.9\paperheight}}{output_17_0.png}
    \end{center}
    { \hspace*{\fill} \\}
    
    Determine and print the value of \texttt{freq0} that achieves the
minimum \texttt{mse1}. Also, plot the estimated function
\texttt{audio\_fn.yhati} for that \texttt{freq0} along with the original
samples \texttt{yi}.

    \begin{Verbatim}[commandchars=\\\{\}]
{\color{incolor}In [{\color{incolor}9}]:} \PY{c+c1}{\PYZsh{} TODO}
        \PY{n}{min\PYZus{}freq0} \PY{o}{=} \PY{n}{freq0}\PY{p}{[}\PY{n}{np}\PY{o}{.}\PY{n}{argmin}\PY{p}{(}\PY{n}{mse1}\PY{p}{)}\PY{p}{]}
        \PY{n+nb}{print}\PY{p}{(}\PY{l+s+s1}{\PYZsq{}}\PY{l+s+s1}{freq0 that minimums mse1: }\PY{l+s+si}{\PYZpc{}.6f}\PY{l+s+s1}{Hz}\PY{l+s+s1}{\PYZsq{}} \PY{o}{\PYZpc{}} \PY{n}{min\PYZus{}freq0}\PY{p}{)}
        \PY{n}{audio\PYZus{}fn}\PY{o}{.}\PY{n}{feval}\PY{p}{(}\PY{n}{min\PYZus{}freq0}\PY{p}{)}
        \PY{n}{plt}\PY{o}{.}\PY{n}{plot}\PY{p}{(}\PY{n}{np}\PY{o}{.}\PY{n}{array}\PY{p}{(}\PY{n+nb}{range}\PY{p}{(}\PY{n+nb}{len}\PY{p}{(}\PY{n}{yi}\PY{p}{)}\PY{p}{)}\PY{p}{)}\PY{o}{/}\PY{n}{sr}\PY{o}{*}\PY{l+m+mi}{1000}\PY{p}{,} \PYZbs{}
                 \PY{n}{yi}\PY{p}{,} \PY{n}{label}\PY{o}{=}\PY{l+s+s1}{\PYZsq{}}\PY{l+s+s1}{Original samples}\PY{l+s+s1}{\PYZsq{}}\PY{p}{)}
        \PY{n}{plt}\PY{o}{.}\PY{n}{plot}\PY{p}{(}\PY{n}{np}\PY{o}{.}\PY{n}{array}\PY{p}{(}\PY{n+nb}{range}\PY{p}{(}\PY{n+nb}{len}\PY{p}{(}\PY{n}{yi}\PY{p}{)}\PY{p}{)}\PY{p}{)}\PY{o}{/}\PY{n}{sr}\PY{o}{*}\PY{l+m+mi}{1000}\PY{p}{,} \PYZbs{}
                 \PY{n}{audio\PYZus{}fn}\PY{o}{.}\PY{n}{yhati}\PY{p}{,} \PY{n}{label}\PY{o}{=}\PY{l+s+s1}{\PYZsq{}}\PY{l+s+s1}{Estimated samples}\PY{l+s+s1}{\PYZsq{}}\PY{p}{)}
        \PY{n}{plt}\PY{o}{.}\PY{n}{xlabel}\PY{p}{(}\PY{l+s+s1}{\PYZsq{}}\PY{l+s+s1}{Time(ms)}\PY{l+s+s1}{\PYZsq{}}\PY{p}{)}
        \PY{n}{plt}\PY{o}{.}\PY{n}{ylabel}\PY{p}{(}\PY{l+s+s1}{\PYZsq{}}\PY{l+s+s1}{Amplitude}\PY{l+s+s1}{\PYZsq{}}\PY{p}{)}
        \PY{n}{plt}\PY{o}{.}\PY{n}{legend}\PY{p}{(}\PY{p}{)}
        \PY{n}{plt}\PY{o}{.}\PY{n}{show}\PY{p}{(}\PY{p}{)}
\end{Verbatim}


    \begin{Verbatim}[commandchars=\\\{\}]
freq0 that minimums mse1: 132.929293Hz

    \end{Verbatim}

    \begin{center}
    \adjustimage{max size={0.9\linewidth}{0.9\paperheight}}{output_19_1.png}
    \end{center}
    { \hspace*{\fill} \\}
    
    \subsection{Computing the Gradient}\label{computing-the-gradient}

The above method found the estimate for \texttt{freq0} by performing a
search over 100 different frequency values and selecting the frequency
value with the lowest MSE. We now see if we can estimate the frequency
with gradient descent minimization of the MSE. We first need to modify
the \texttt{feval} method in the \texttt{AudioFitFn} class above to
compute the gradient. Some elementary calculus (see the homework), shows
that

\begin{verbatim}
dmse1(freq0)/dfreq0 = dmse(freq0,betahat)/dfreq0
\end{verbatim}

So, we just need to evaluate the partial derivative of
\texttt{mse\ =\ np.mean((yi-yhati)**2)} with respect to the parameter
\texttt{freq0} holding the parameters \texttt{beta=betahat}. Modify the
\texttt{feval} method above to compute the gradient and return the
gradient in \texttt{mse1\_grad}.

    \begin{Verbatim}[commandchars=\\\{\}]
{\color{incolor}In [{\color{incolor}10}]:} \PY{k}{class} \PY{n+nc}{AudioFitFn}\PY{p}{(}\PY{n+nb}{object}\PY{p}{)}\PY{p}{:}
             \PY{k}{def} \PY{n+nf}{\PYZus{}\PYZus{}init\PYZus{}\PYZus{}}\PY{p}{(}\PY{n+nb+bp}{self}\PY{p}{,}\PY{n}{yi}\PY{p}{,}\PY{n}{sr}\PY{o}{=}\PY{l+m+mi}{44100}\PY{p}{,}\PY{n}{nterms}\PY{o}{=}\PY{l+m+mi}{8}\PY{p}{)}\PY{p}{:}
                 \PY{l+s+sd}{\PYZdq{}\PYZdq{}\PYZdq{}}
         \PY{l+s+sd}{        A class for fitting }
         \PY{l+s+sd}{        }
         \PY{l+s+sd}{        yi:  One frame of audio}
         \PY{l+s+sd}{        sr:  Sample rate (in Hz)}
         \PY{l+s+sd}{        nterms:  Number of harmonics used in the model (default=8)}
         \PY{l+s+sd}{        \PYZdq{}\PYZdq{}\PYZdq{}}
                 \PY{n+nb+bp}{self}\PY{o}{.}\PY{n}{yi} \PY{o}{=} \PY{n}{yi}
                 \PY{n+nb+bp}{self}\PY{o}{.}\PY{n}{sr} \PY{o}{=} \PY{n}{sr}
                 \PY{n+nb+bp}{self}\PY{o}{.}\PY{n}{nterms} \PY{o}{=} \PY{n}{nterms}
                         
             \PY{k}{def} \PY{n+nf}{feval}\PY{p}{(}\PY{n+nb+bp}{self}\PY{p}{,}\PY{n}{freq0}\PY{p}{)}\PY{p}{:}
                 \PY{l+s+sd}{\PYZdq{}\PYZdq{}\PYZdq{}}
         \PY{l+s+sd}{        Optimization function for audio fitting.  Given a fundamental frequency, freq0, the }
         \PY{l+s+sd}{        method performs a least squares fit for the audio sample using the model:}
         \PY{l+s+sd}{        }
         \PY{l+s+sd}{        yhati[k] = c + \PYZbs{}sum\PYZus{}\PYZob{}j=0\PYZcb{}\PYZca{}\PYZob{}nterms\PYZhy{}1\PYZcb{} a[j]*cos(2*np.pi*k*freq0*(j+1)/sr) }
         \PY{l+s+sd}{                                          +  b[j]*sin(2*np.pi*k*freq0*(j+1)/sr)}
         \PY{l+s+sd}{        }
         \PY{l+s+sd}{        The coefficients beta = [c,a[0],...,a[nterms\PYZhy{}1],b[0],...,b[nterms\PYZhy{}1]] }
         \PY{l+s+sd}{        are found by least squares.}
         
         \PY{l+s+sd}{        Returns:}
         \PY{l+s+sd}{        }
         \PY{l+s+sd}{        mse1:   The MSE of the best least square fit.}
         \PY{l+s+sd}{        mse1\PYZus{}grad:  The gradient of mse1 wrt to the parameter freq0}
         \PY{l+s+sd}{        \PYZdq{}\PYZdq{}\PYZdq{}}
                 
                 \PY{c+c1}{\PYZsh{} TODO   }
                 \PY{n}{cos\PYZus{}vals} \PY{o}{=} \PY{p}{[}\PY{p}{[}\PY{n}{np}\PY{o}{.}\PY{n}{cos}\PY{p}{(}\PY{l+m+mi}{2}\PY{o}{*}\PY{n}{np}\PY{o}{.}\PY{n}{pi}\PY{o}{*}\PY{n}{k}\PY{o}{*}\PY{n}{freq0}\PY{o}{*}\PY{p}{(}\PY{n}{j}\PY{o}{+}\PY{l+m+mi}{1}\PY{p}{)}\PY{o}{/}\PY{n+nb+bp}{self}\PY{o}{.}\PY{n}{sr}\PY{p}{)} \PYZbs{}
                              \PY{k}{for} \PY{n}{j} \PY{o+ow}{in} \PY{n+nb}{range}\PY{p}{(}\PY{n+nb+bp}{self}\PY{o}{.}\PY{n}{nterms}\PY{p}{)}\PY{p}{]} \PY{k}{for} \PY{n}{k} \PY{o+ow}{in} \PY{n+nb}{range}\PY{p}{(}\PY{n+nb}{len}\PY{p}{(}\PY{n+nb+bp}{self}\PY{o}{.}\PY{n}{yi}\PY{p}{)}\PY{p}{)}\PY{p}{]}
                 \PY{n}{sin\PYZus{}vals} \PY{o}{=} \PY{p}{[}\PY{p}{[}\PY{n}{np}\PY{o}{.}\PY{n}{sin}\PY{p}{(}\PY{l+m+mi}{2}\PY{o}{*}\PY{n}{np}\PY{o}{.}\PY{n}{pi}\PY{o}{*}\PY{n}{k}\PY{o}{*}\PY{n}{freq0}\PY{o}{*}\PY{p}{(}\PY{n}{j}\PY{o}{+}\PY{l+m+mi}{1}\PY{p}{)}\PY{o}{/}\PY{n+nb+bp}{self}\PY{o}{.}\PY{n}{sr}\PY{p}{)} \PYZbs{}
                              \PY{k}{for} \PY{n}{j} \PY{o+ow}{in} \PY{n+nb}{range}\PY{p}{(}\PY{n+nb+bp}{self}\PY{o}{.}\PY{n}{nterms}\PY{p}{)}\PY{p}{]} \PY{k}{for} \PY{n}{k} \PY{o+ow}{in} \PY{n+nb}{range}\PY{p}{(}\PY{n+nb}{len}\PY{p}{(}\PY{n+nb+bp}{self}\PY{o}{.}\PY{n}{yi}\PY{p}{)}\PY{p}{)}\PY{p}{]}
                 \PY{n}{A} \PY{o}{=} \PY{n}{np}\PY{o}{.}\PY{n}{column\PYZus{}stack}\PY{p}{(}\PY{p}{(}\PY{n}{np}\PY{o}{.}\PY{n}{ones}\PY{p}{(}\PY{n+nb}{len}\PY{p}{(}\PY{n+nb+bp}{self}\PY{o}{.}\PY{n}{yi}\PY{p}{)}\PY{p}{,}\PY{p}{)}\PY{p}{,} \PY{n}{cos\PYZus{}vals}\PY{p}{,} \PY{n}{sin\PYZus{}vals}\PY{p}{)}\PY{p}{)}
                 \PY{n}{beta} \PY{o}{=} \PY{n}{np}\PY{o}{.}\PY{n}{linalg}\PY{o}{.}\PY{n}{lstsq}\PY{p}{(}\PY{n}{A}\PY{p}{,} \PY{n+nb+bp}{self}\PY{o}{.}\PY{n}{yi}\PY{p}{,} \PY{n}{rcond}\PY{o}{=}\PY{k+kc}{None}\PY{p}{)}
                 \PY{n+nb+bp}{self}\PY{o}{.}\PY{n}{yhati} \PY{o}{=} \PY{n}{A}\PY{o}{.}\PY{n}{dot}\PY{p}{(}\PY{n}{beta}\PY{p}{[}\PY{l+m+mi}{0}\PY{p}{]}\PY{p}{)}
                 \PY{n}{mse1} \PY{o}{=} \PY{n}{np}\PY{o}{.}\PY{n}{mean}\PY{p}{(}\PY{p}{(}\PY{n+nb+bp}{self}\PY{o}{.}\PY{n}{yi}\PY{o}{\PYZhy{}}\PY{n+nb+bp}{self}\PY{o}{.}\PY{n}{yhati}\PY{p}{)}\PY{o}{*}\PY{o}{*}\PY{l+m+mi}{2}\PY{p}{)}
                 
                 \PY{c+c1}{\PYZsh{} Compute the gradient wrt to freq0 }
                 
                 \PY{c+c1}{\PYZsh{} TODO }
                 \PY{n}{dcos\PYZus{}vals} \PY{o}{=} \PY{p}{[}\PY{p}{[}\PY{n}{np}\PY{o}{.}\PY{n}{sin}\PY{p}{(}\PY{o}{\PYZhy{}}\PY{l+m+mi}{2}\PY{o}{*}\PY{n}{np}\PY{o}{.}\PY{n}{pi}\PY{o}{*}\PY{n}{k}\PY{o}{*}\PY{n}{freq0}\PY{o}{*}\PY{p}{(}\PY{n}{j}\PY{o}{+}\PY{l+m+mi}{1}\PY{p}{)}\PY{o}{/}\PY{n+nb+bp}{self}\PY{o}{.}\PY{n}{sr}\PY{p}{)}\PY{o}{*}\PY{p}{(}\PY{l+m+mi}{2}\PY{o}{*}\PY{n}{np}\PY{o}{.}\PY{n}{pi}\PY{o}{*}\PY{n}{k}\PY{o}{*}\PY{p}{(}\PY{n}{j}\PY{o}{+}\PY{l+m+mi}{1}\PY{p}{)}\PY{o}{/}\PY{n+nb+bp}{self}\PY{o}{.}\PY{n}{sr}\PY{p}{)} \PYZbs{}
                               \PY{k}{for} \PY{n}{j} \PY{o+ow}{in} \PY{n+nb}{range}\PY{p}{(}\PY{n+nb+bp}{self}\PY{o}{.}\PY{n}{nterms}\PY{p}{)}\PY{p}{]} \PY{k}{for} \PY{n}{k} \PY{o+ow}{in} \PY{n+nb}{range}\PY{p}{(}\PY{n+nb}{len}\PY{p}{(}\PY{n+nb+bp}{self}\PY{o}{.}\PY{n}{yi}\PY{p}{)}\PY{p}{)}\PY{p}{]}
                 \PY{n}{dsin\PYZus{}vals} \PY{o}{=} \PY{p}{[}\PY{p}{[}\PY{n}{np}\PY{o}{.}\PY{n}{cos}\PY{p}{(}\PY{l+m+mi}{2}\PY{o}{*}\PY{n}{np}\PY{o}{.}\PY{n}{pi}\PY{o}{*}\PY{n}{k}\PY{o}{*}\PY{n}{freq0}\PY{o}{*}\PY{p}{(}\PY{n}{j}\PY{o}{+}\PY{l+m+mi}{1}\PY{p}{)}\PY{o}{/}\PY{n+nb+bp}{self}\PY{o}{.}\PY{n}{sr}\PY{p}{)}\PY{o}{*}\PY{p}{(}\PY{l+m+mi}{2}\PY{o}{*}\PY{n}{np}\PY{o}{.}\PY{n}{pi}\PY{o}{*}\PY{n}{k}\PY{o}{*}\PY{p}{(}\PY{n}{j}\PY{o}{+}\PY{l+m+mi}{1}\PY{p}{)}\PY{o}{/}\PY{n+nb+bp}{self}\PY{o}{.}\PY{n}{sr}\PY{p}{)} \PYZbs{}
                               \PY{k}{for} \PY{n}{j} \PY{o+ow}{in} \PY{n+nb}{range}\PY{p}{(}\PY{n+nb+bp}{self}\PY{o}{.}\PY{n}{nterms}\PY{p}{)}\PY{p}{]} \PY{k}{for} \PY{n}{k} \PY{o+ow}{in} \PY{n+nb}{range}\PY{p}{(}\PY{n+nb}{len}\PY{p}{(}\PY{n+nb+bp}{self}\PY{o}{.}\PY{n}{yi}\PY{p}{)}\PY{p}{)}\PY{p}{]}
                 \PY{n}{A\PYZus{}grad} \PY{o}{=} \PY{n}{np}\PY{o}{.}\PY{n}{column\PYZus{}stack}\PY{p}{(}\PY{p}{(}\PY{n}{np}\PY{o}{.}\PY{n}{zeros}\PY{p}{(}\PY{n+nb}{len}\PY{p}{(}\PY{n+nb+bp}{self}\PY{o}{.}\PY{n}{yi}\PY{p}{)}\PY{p}{,}\PY{p}{)}\PY{p}{,} \PY{n}{dcos\PYZus{}vals}\PY{p}{,} \PY{n}{dsin\PYZus{}vals}\PY{p}{)}\PY{p}{)}
                 \PY{n+nb+bp}{self}\PY{o}{.}\PY{n}{yhati\PYZus{}grad} \PY{o}{=} \PY{n}{A\PYZus{}grad}\PY{o}{.}\PY{n}{dot}\PY{p}{(}\PY{n}{beta}\PY{p}{[}\PY{l+m+mi}{0}\PY{p}{]}\PY{p}{)}
                 \PY{n}{mse1\PYZus{}grad} \PY{o}{=} \PY{n}{np}\PY{o}{.}\PY{n}{mean}\PY{p}{(}\PY{l+m+mi}{2}\PY{o}{*}\PY{p}{(}\PY{n+nb+bp}{self}\PY{o}{.}\PY{n}{yhati}\PY{o}{\PYZhy{}}\PY{n+nb+bp}{self}\PY{o}{.}\PY{n}{yi}\PY{p}{)}\PY{o}{*}\PY{n+nb+bp}{self}\PY{o}{.}\PY{n}{yhati\PYZus{}grad}\PY{p}{)}
                 \PY{k}{return} \PY{n}{mse1}\PY{p}{,} \PY{n}{mse1\PYZus{}grad}
\end{Verbatim}


    Now, test the gradient by taking two close values of \texttt{freq0}, say
\texttt{freq0\_0} and \texttt{freq0\_1} and verifying that first-order
approximation holds.

    \begin{Verbatim}[commandchars=\\\{\}]
{\color{incolor}In [{\color{incolor}11}]:} \PY{c+c1}{\PYZsh{} TODO}
         \PY{n}{audio\PYZus{}fn} \PY{o}{=} \PY{n}{AudioFitFn}\PY{p}{(}\PY{n}{yi}\PY{p}{,} \PY{n}{sr}\PY{p}{)}
         
         \PY{n}{freq0\PYZus{}0} \PY{o}{=} \PY{n}{min\PYZus{}freq0}
         \PY{n}{freq0\PYZus{}1} \PY{o}{=} \PY{n}{freq0\PYZus{}0} \PY{o}{+} \PY{l+m+mf}{1e\PYZhy{}6}
         
         \PY{n}{mse1\PYZus{}0}\PY{p}{,} \PY{n}{mse1\PYZus{}grad\PYZus{}0} \PY{o}{=} \PY{n}{audio\PYZus{}fn}\PY{o}{.}\PY{n}{feval}\PY{p}{(}\PY{n}{freq0\PYZus{}0}\PY{p}{)}
         \PY{n}{mse1\PYZus{}1}\PY{p}{,} \PY{n}{\PYZus{}}           \PY{o}{=} \PY{n}{audio\PYZus{}fn}\PY{o}{.}\PY{n}{feval}\PY{p}{(}\PY{n}{freq0\PYZus{}1}\PY{p}{)}
         
         \PY{n+nb}{print}\PY{p}{(}\PY{l+s+s1}{\PYZsq{}}\PY{l+s+s1}{LHS = }\PY{l+s+si}{\PYZpc{}.12e}\PY{l+s+s1}{\PYZsq{}} \PY{o}{\PYZpc{}} \PY{p}{(}\PY{n}{mse1\PYZus{}0}\PY{o}{\PYZhy{}}\PY{n}{mse1\PYZus{}1}\PY{p}{)}\PY{p}{)}
         \PY{n+nb}{print}\PY{p}{(}\PY{l+s+s1}{\PYZsq{}}\PY{l+s+s1}{RHS = }\PY{l+s+si}{\PYZpc{}.12e}\PY{l+s+s1}{\PYZsq{}} \PY{o}{\PYZpc{}} \PY{p}{(}\PY{p}{(}\PY{n}{mse1\PYZus{}grad\PYZus{}0}\PY{p}{)} \PY{o}{*} \PY{p}{(}\PY{n}{freq0\PYZus{}0}\PY{o}{\PYZhy{}}\PY{n}{freq0\PYZus{}1}\PY{p}{)}\PY{p}{)}\PY{p}{)}
\end{Verbatim}


    \begin{Verbatim}[commandchars=\\\{\}]
LHS = -1.045718676929e-07
RHS = -1.045718388664e-07

    \end{Verbatim}

    \subsection{Run the Optimizer}\label{run-the-optimizer}

    We cut and paste the optimizer from the
\href{./grad_descent.ipynb}{gradient descent demo}.

    \begin{Verbatim}[commandchars=\\\{\}]
{\color{incolor}In [{\color{incolor}12}]:} \PY{k}{def} \PY{n+nf}{grad\PYZus{}opt\PYZus{}adapt}\PY{p}{(}\PY{n}{feval}\PY{p}{,} \PY{n}{winit}\PY{p}{,} \PY{n}{nit}\PY{o}{=}\PY{l+m+mi}{1000}\PY{p}{,} \PY{n}{lr\PYZus{}init}\PY{o}{=}\PY{l+m+mf}{1e\PYZhy{}3}\PY{p}{)}\PY{p}{:}
             \PY{l+s+sd}{\PYZdq{}\PYZdq{}\PYZdq{}}
         \PY{l+s+sd}{    Gradient descent optimization with adaptive step size}
         \PY{l+s+sd}{    }
         \PY{l+s+sd}{    feval:  A function that returns f, fgrad, the objective}
         \PY{l+s+sd}{            function and its gradient}
         \PY{l+s+sd}{    winit:  Initial estimate}
         \PY{l+s+sd}{    nit:    Number of iterations}
         \PY{l+s+sd}{    lr:     Initial learning rate}
         \PY{l+s+sd}{    }
         \PY{l+s+sd}{    Returns:}
         \PY{l+s+sd}{    w:   Final estimate for the optimal}
         \PY{l+s+sd}{    f0:  Function at the optimal}
         \PY{l+s+sd}{    \PYZdq{}\PYZdq{}\PYZdq{}}
             
             \PY{c+c1}{\PYZsh{} Set initial point}
             \PY{n}{w0} \PY{o}{=} \PY{n}{winit}
             \PY{n}{f0}\PY{p}{,} \PY{n}{fgrad0} \PY{o}{=} \PY{n}{feval}\PY{p}{(}\PY{n}{w0}\PY{p}{)}
             \PY{n}{lr} \PY{o}{=} \PY{n}{lr\PYZus{}init}
             
             \PY{c+c1}{\PYZsh{} Create history dictionary for tracking progress per iteration.}
             \PY{c+c1}{\PYZsh{} This isn\PYZsq{}t necessary if you just want the final answer, but it }
             \PY{c+c1}{\PYZsh{} is useful for debugging}
             \PY{n}{hist} \PY{o}{=} \PY{p}{\PYZob{}}\PY{l+s+s1}{\PYZsq{}}\PY{l+s+s1}{lr}\PY{l+s+s1}{\PYZsq{}}\PY{p}{:} \PY{p}{[}\PY{p}{]}\PY{p}{,} \PY{l+s+s1}{\PYZsq{}}\PY{l+s+s1}{w}\PY{l+s+s1}{\PYZsq{}}\PY{p}{:} \PY{p}{[}\PY{p}{]}\PY{p}{,} \PY{l+s+s1}{\PYZsq{}}\PY{l+s+s1}{f}\PY{l+s+s1}{\PYZsq{}}\PY{p}{:} \PY{p}{[}\PY{p}{]}\PY{p}{\PYZcb{}}
         
             \PY{k}{for} \PY{n}{it} \PY{o+ow}{in} \PY{n+nb}{range}\PY{p}{(}\PY{n}{nit}\PY{p}{)}\PY{p}{:}
         
                 \PY{c+c1}{\PYZsh{} Take a gradient step}
                 \PY{n}{w1} \PY{o}{=} \PY{n}{w0} \PY{o}{\PYZhy{}} \PY{n}{lr}\PY{o}{*}\PY{n}{fgrad0}
         
                 \PY{c+c1}{\PYZsh{} Evaluate the test point by computing the objective function, f1,}
                 \PY{c+c1}{\PYZsh{} at the test point and the predicted decrease, df\PYZus{}est}
                 \PY{n}{f1}\PY{p}{,} \PY{n}{fgrad1} \PY{o}{=} \PY{n}{feval}\PY{p}{(}\PY{n}{w1}\PY{p}{)}
                 \PY{n}{df\PYZus{}est} \PY{o}{=} \PY{n}{np}\PY{o}{.}\PY{n}{dot}\PY{p}{(}\PY{n}{fgrad0}\PY{p}{,} \PY{p}{(}\PY{n}{w1}\PY{o}{\PYZhy{}}\PY{n}{w0}\PY{p}{)}\PY{p}{)}
                 
                 \PY{c+c1}{\PYZsh{} Check if test point passes the Armijo rule}
                 \PY{n}{alpha} \PY{o}{=} \PY{l+m+mf}{0.5}
                 \PY{k}{if} \PY{p}{(}\PY{n}{f1}\PY{o}{\PYZhy{}}\PY{n}{f0} \PY{o}{\PYZlt{}} \PY{n}{alpha}\PY{o}{*}\PY{n}{df\PYZus{}est}\PY{p}{)} \PY{o+ow}{and} \PY{p}{(}\PY{n}{f1} \PY{o}{\PYZlt{}} \PY{n}{f0}\PY{p}{)}\PY{p}{:}
                     \PY{c+c1}{\PYZsh{} If descent is sufficient, accept the point and increase the}
                     \PY{c+c1}{\PYZsh{} learning rate}
                     \PY{n}{lr} \PY{o}{=} \PY{n}{lr}\PY{o}{*}\PY{l+m+mi}{2}
                     \PY{n}{f0} \PY{o}{=} \PY{n}{f1}
                     \PY{n}{fgrad0} \PY{o}{=} \PY{n}{fgrad1}
                     \PY{n}{w0} \PY{o}{=} \PY{n}{w1}
                 \PY{k}{else}\PY{p}{:}
                     \PY{c+c1}{\PYZsh{} Otherwise, decrease the learning rate}
                     \PY{n}{lr} \PY{o}{=} \PY{n}{lr}\PY{o}{/}\PY{l+m+mi}{2}            
                     
                 \PY{c+c1}{\PYZsh{} Save history}
                 \PY{n}{hist}\PY{p}{[}\PY{l+s+s1}{\PYZsq{}}\PY{l+s+s1}{f}\PY{l+s+s1}{\PYZsq{}}\PY{p}{]}\PY{o}{.}\PY{n}{append}\PY{p}{(}\PY{n}{f0}\PY{p}{)}
                 \PY{n}{hist}\PY{p}{[}\PY{l+s+s1}{\PYZsq{}}\PY{l+s+s1}{lr}\PY{l+s+s1}{\PYZsq{}}\PY{p}{]}\PY{o}{.}\PY{n}{append}\PY{p}{(}\PY{n}{lr}\PY{p}{)}
                 \PY{n}{hist}\PY{p}{[}\PY{l+s+s1}{\PYZsq{}}\PY{l+s+s1}{w}\PY{l+s+s1}{\PYZsq{}}\PY{p}{]}\PY{o}{.}\PY{n}{append}\PY{p}{(}\PY{n}{w0}\PY{p}{)}
         
             \PY{c+c1}{\PYZsh{} Convert to numpy arrays}
             \PY{k}{for} \PY{n}{elem} \PY{o+ow}{in} \PY{p}{(}\PY{l+s+s1}{\PYZsq{}}\PY{l+s+s1}{f}\PY{l+s+s1}{\PYZsq{}}\PY{p}{,} \PY{l+s+s1}{\PYZsq{}}\PY{l+s+s1}{lr}\PY{l+s+s1}{\PYZsq{}}\PY{p}{,} \PY{l+s+s1}{\PYZsq{}}\PY{l+s+s1}{w}\PY{l+s+s1}{\PYZsq{}}\PY{p}{)}\PY{p}{:}
                 \PY{n}{hist}\PY{p}{[}\PY{n}{elem}\PY{p}{]} \PY{o}{=} \PY{n}{np}\PY{o}{.}\PY{n}{array}\PY{p}{(}\PY{n}{hist}\PY{p}{[}\PY{n}{elem}\PY{p}{]}\PY{p}{)}
             \PY{k}{return} \PY{n}{w0}\PY{p}{,} \PY{n}{f0}\PY{p}{,} \PY{n}{hist}
\end{Verbatim}


    Now, run the optimizer with the feval function with a starting estimate
for freq0 = 130 Hz. Use \texttt{lr\_init=1e-3} and
\texttt{f0\_init=130}. Print the final frequency estimate. Also, print
the \href{https://newt.phys.unsw.edu.au/jw/notes.html}{midi number} of
the estimated frequency:

\begin{verbatim}
 midi_num = 12*log2(freq/440 Hz) + 69
 
\end{verbatim}

If the note was exactly a musical note, \texttt{midi\_num} should be an
integer. But you will see that the frequency does not exactly lie on a
note since the pitch in a viola bends around the note.

    \begin{Verbatim}[commandchars=\\\{\}]
{\color{incolor}In [{\color{incolor}13}]:} \PY{c+c1}{\PYZsh{} TODO}
         \PY{n}{lr\PYZus{}init} \PY{o}{=} \PY{l+m+mf}{1e\PYZhy{}3}
         \PY{n}{f0\PYZus{}init} \PY{o}{=} \PY{l+m+mi}{130}
         \PY{n}{freq0\PYZus{}min}\PY{p}{,} \PY{n}{mse\PYZus{}min}\PY{p}{,} \PY{n}{hist} \PY{o}{=} \PY{n}{grad\PYZus{}opt\PYZus{}adapt}\PY{p}{(}\PY{n}{audio\PYZus{}fn}\PY{o}{.}\PY{n}{feval}\PY{p}{,} \PY{n}{winit}\PY{o}{=}\PY{n}{f0\PYZus{}init}\PY{p}{,} \PY{n}{lr\PYZus{}init}\PY{o}{=}\PY{n}{lr\PYZus{}init}\PY{p}{)}
         
         \PY{n}{midi\PYZus{}num} \PY{o}{=} \PY{l+m+mi}{12} \PY{o}{*} \PY{n}{np}\PY{o}{.}\PY{n}{log2}\PY{p}{(}\PY{n}{freq0\PYZus{}min} \PY{o}{/} \PY{l+m+mi}{440}\PY{p}{)} \PY{o}{+} \PY{l+m+mi}{69}
         
         \PY{n+nb}{print}\PY{p}{(}\PY{l+s+s2}{\PYZdq{}}\PY{l+s+s2}{f0\PYZus{}min=}\PY{l+s+si}{\PYZpc{}f}\PY{l+s+s2}{ midi=}\PY{l+s+si}{\PYZpc{}f}\PY{l+s+s2}{\PYZdq{}} \PY{o}{\PYZpc{}} \PY{p}{(}\PY{n}{freq0\PYZus{}min}\PY{p}{,} \PY{n}{midi\PYZus{}num}\PY{p}{)}\PY{p}{)}
\end{Verbatim}


    \begin{Verbatim}[commandchars=\\\{\}]
f0\_min=131.528923 midi=48.094519

    \end{Verbatim}

    Plot the MSE as a function of the iteration. You should tell whether you
gradient algorithm has converged. If not, you may need to adjust the
number of iterations.

    \begin{Verbatim}[commandchars=\\\{\}]
{\color{incolor}In [{\color{incolor}14}]:} \PY{c+c1}{\PYZsh{}TODO}
         \PY{n}{plt}\PY{o}{.}\PY{n}{semilogx}\PY{p}{(}\PY{n}{hist}\PY{p}{[}\PY{l+s+s1}{\PYZsq{}}\PY{l+s+s1}{f}\PY{l+s+s1}{\PYZsq{}}\PY{p}{]}\PY{p}{)}
         \PY{n}{plt}\PY{o}{.}\PY{n}{xlabel}\PY{p}{(}\PY{l+s+s1}{\PYZsq{}}\PY{l+s+s1}{Iterations}\PY{l+s+s1}{\PYZsq{}}\PY{p}{)}
         \PY{n}{plt}\PY{o}{.}\PY{n}{ylabel}\PY{p}{(}\PY{l+s+s1}{\PYZsq{}}\PY{l+s+s1}{MSE}\PY{l+s+s1}{\PYZsq{}}\PY{p}{)}
         \PY{n}{plt}\PY{o}{.}\PY{n}{show}\PY{p}{(}\PY{p}{)}
\end{Verbatim}


    \begin{center}
    \adjustimage{max size={0.9\linewidth}{0.9\paperheight}}{output_30_0.png}
    \end{center}
    { \hspace*{\fill} \\}
    
    Compare your solution of f0\_min with the one obtained using exhaustive
search. You may notice a slight difference. Explain why and also which
solution is likely to be correct?

    Answer:\\
This is because the samples of frequency are discrete. That is, if the
number of freq0 is infinity, and the step size is extremely near 0, we
should get the same results with these two method. However, in this
case, the latter one is more likely to be correct, since it has a small
step size, which corresponds to a higher accuracy.

    Now, repeat with an initial frequency of 200 Hz. Print the final
estimated frequency. Also plot the MSE per iteration on the same graph
as the MSE per iteration with the initial condition = 130 Hz.

    \begin{Verbatim}[commandchars=\\\{\}]
{\color{incolor}In [{\color{incolor}15}]:} \PY{c+c1}{\PYZsh{} TODO}
         \PY{n}{freq0\PYZus{}min1}\PY{p}{,} \PY{n}{mse\PYZus{}min1}\PY{p}{,} \PY{n}{hist1} \PY{o}{=} \PY{n}{grad\PYZus{}opt\PYZus{}adapt}\PY{p}{(}\PY{n}{audio\PYZus{}fn}\PY{o}{.}\PY{n}{feval}\PY{p}{,} \PY{n}{winit}\PY{o}{=}\PY{l+m+mi}{200}\PY{p}{,} \PY{n}{lr\PYZus{}init}\PY{o}{=}\PY{l+m+mf}{1e\PYZhy{}3}\PY{p}{)}
         \PY{n}{plt}\PY{o}{.}\PY{n}{semilogx}\PY{p}{(}\PY{n}{hist}\PY{p}{[}\PY{l+s+s1}{\PYZsq{}}\PY{l+s+s1}{f}\PY{l+s+s1}{\PYZsq{}}\PY{p}{]}\PY{p}{)}
         \PY{n}{plt}\PY{o}{.}\PY{n}{semilogx}\PY{p}{(}\PY{n}{hist1}\PY{p}{[}\PY{l+s+s1}{\PYZsq{}}\PY{l+s+s1}{f}\PY{l+s+s1}{\PYZsq{}}\PY{p}{]}\PY{p}{)}
         \PY{n}{plt}\PY{o}{.}\PY{n}{xlabel}\PY{p}{(}\PY{l+s+s1}{\PYZsq{}}\PY{l+s+s1}{Iterations}\PY{l+s+s1}{\PYZsq{}}\PY{p}{)}
         \PY{n}{plt}\PY{o}{.}\PY{n}{ylabel}\PY{p}{(}\PY{l+s+s1}{\PYZsq{}}\PY{l+s+s1}{MSE}\PY{l+s+s1}{\PYZsq{}}\PY{p}{)}
         \PY{n+nb}{print}\PY{p}{(}\PY{l+s+s1}{\PYZsq{}}\PY{l+s+s1}{Final estimated frequency: }\PY{l+s+si}{\PYZpc{}f}\PY{l+s+s1}{\PYZsq{}} \PY{o}{\PYZpc{}} \PY{n}{freq0\PYZus{}min1}\PY{p}{)}
\end{Verbatim}


    \begin{Verbatim}[commandchars=\\\{\}]
Final estimated frequency: 197.872343

    \end{Verbatim}

    \begin{center}
    \adjustimage{max size={0.9\linewidth}{0.9\paperheight}}{output_34_1.png}
    \end{center}
    { \hspace*{\fill} \\}
    
    Did you get the same solution as when you used an initial solution of
130 Hz? Why?

    Answer:\\
No. The optimizer does not reach the real minimum MSE, since the initial
frequency led to another local minimum. So a good initial value is
required if a good result is expected.

    \subsection{More Fun}\label{more-fun}

While the above method does not work very well, there are many good
approaches. For one thing, we can obtain a good initial condition using
an FFT of the frame. The FFT is used in many pitch detection methods.
More difficult problems include multi-tone detection, chord detection
and instrument separation. A useful python library that contains all
sorts of interesting audio analysis tools in the
\href{https://librosa.github.io/librosa/}{librosa package}.


    % Add a bibliography block to the postdoc
    
    
    
    \end{document}
