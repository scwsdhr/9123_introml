
% Default to the notebook output style

    


% Inherit from the specified cell style.




    
\documentclass[11pt]{article}

    
    
    \usepackage[T1]{fontenc}
    % Nicer default font (+ math font) than Computer Modern for most use cases
    \usepackage{mathpazo}

    % Basic figure setup, for now with no caption control since it's done
    % automatically by Pandoc (which extracts ![](path) syntax from Markdown).
    \usepackage{graphicx}
    % We will generate all images so they have a width \maxwidth. This means
    % that they will get their normal width if they fit onto the page, but
    % are scaled down if they would overflow the margins.
    \makeatletter
    \def\maxwidth{\ifdim\Gin@nat@width>\linewidth\linewidth
    \else\Gin@nat@width\fi}
    \makeatother
    \let\Oldincludegraphics\includegraphics
    % Set max figure width to be 80% of text width, for now hardcoded.
    \renewcommand{\includegraphics}[1]{\Oldincludegraphics[width=.8\maxwidth]{#1}}
    % Ensure that by default, figures have no caption (until we provide a
    % proper Figure object with a Caption API and a way to capture that
    % in the conversion process - todo).
    \usepackage{caption}
    \DeclareCaptionLabelFormat{nolabel}{}
    \captionsetup{labelformat=nolabel}

    \usepackage{adjustbox} % Used to constrain images to a maximum size 
    \usepackage{xcolor} % Allow colors to be defined
    \usepackage{enumerate} % Needed for markdown enumerations to work
    \usepackage{geometry} % Used to adjust the document margins
    \usepackage{amsmath} % Equations
    \usepackage{amssymb} % Equations
    \usepackage{textcomp} % defines textquotesingle
    % Hack from http://tex.stackexchange.com/a/47451/13684:
    \AtBeginDocument{%
        \def\PYZsq{\textquotesingle}% Upright quotes in Pygmentized code
    }
    \usepackage{upquote} % Upright quotes for verbatim code
    \usepackage{eurosym} % defines \euro
    \usepackage[mathletters]{ucs} % Extended unicode (utf-8) support
    \usepackage[utf8x]{inputenc} % Allow utf-8 characters in the tex document
    \usepackage{fancyvrb} % verbatim replacement that allows latex
    \usepackage{grffile} % extends the file name processing of package graphics 
                         % to support a larger range 
    % The hyperref package gives us a pdf with properly built
    % internal navigation ('pdf bookmarks' for the table of contents,
    % internal cross-reference links, web links for URLs, etc.)
    \usepackage{hyperref}
    \usepackage{longtable} % longtable support required by pandoc >1.10
    \usepackage{booktabs}  % table support for pandoc > 1.12.2
    \usepackage[inline]{enumitem} % IRkernel/repr support (it uses the enumerate* environment)
    \usepackage[normalem]{ulem} % ulem is needed to support strikethroughs (\sout)
                                % normalem makes italics be italics, not underlines
    

    
    
    % Colors for the hyperref package
    \definecolor{urlcolor}{rgb}{0,.145,.698}
    \definecolor{linkcolor}{rgb}{.71,0.21,0.01}
    \definecolor{citecolor}{rgb}{.12,.54,.11}

    % ANSI colors
    \definecolor{ansi-black}{HTML}{3E424D}
    \definecolor{ansi-black-intense}{HTML}{282C36}
    \definecolor{ansi-red}{HTML}{E75C58}
    \definecolor{ansi-red-intense}{HTML}{B22B31}
    \definecolor{ansi-green}{HTML}{00A250}
    \definecolor{ansi-green-intense}{HTML}{007427}
    \definecolor{ansi-yellow}{HTML}{DDB62B}
    \definecolor{ansi-yellow-intense}{HTML}{B27D12}
    \definecolor{ansi-blue}{HTML}{208FFB}
    \definecolor{ansi-blue-intense}{HTML}{0065CA}
    \definecolor{ansi-magenta}{HTML}{D160C4}
    \definecolor{ansi-magenta-intense}{HTML}{A03196}
    \definecolor{ansi-cyan}{HTML}{60C6C8}
    \definecolor{ansi-cyan-intense}{HTML}{258F8F}
    \definecolor{ansi-white}{HTML}{C5C1B4}
    \definecolor{ansi-white-intense}{HTML}{A1A6B2}

    % commands and environments needed by pandoc snippets
    % extracted from the output of `pandoc -s`
    \providecommand{\tightlist}{%
      \setlength{\itemsep}{0pt}\setlength{\parskip}{0pt}}
    \DefineVerbatimEnvironment{Highlighting}{Verbatim}{commandchars=\\\{\}}
    % Add ',fontsize=\small' for more characters per line
    \newenvironment{Shaded}{}{}
    \newcommand{\KeywordTok}[1]{\textcolor[rgb]{0.00,0.44,0.13}{\textbf{{#1}}}}
    \newcommand{\DataTypeTok}[1]{\textcolor[rgb]{0.56,0.13,0.00}{{#1}}}
    \newcommand{\DecValTok}[1]{\textcolor[rgb]{0.25,0.63,0.44}{{#1}}}
    \newcommand{\BaseNTok}[1]{\textcolor[rgb]{0.25,0.63,0.44}{{#1}}}
    \newcommand{\FloatTok}[1]{\textcolor[rgb]{0.25,0.63,0.44}{{#1}}}
    \newcommand{\CharTok}[1]{\textcolor[rgb]{0.25,0.44,0.63}{{#1}}}
    \newcommand{\StringTok}[1]{\textcolor[rgb]{0.25,0.44,0.63}{{#1}}}
    \newcommand{\CommentTok}[1]{\textcolor[rgb]{0.38,0.63,0.69}{\textit{{#1}}}}
    \newcommand{\OtherTok}[1]{\textcolor[rgb]{0.00,0.44,0.13}{{#1}}}
    \newcommand{\AlertTok}[1]{\textcolor[rgb]{1.00,0.00,0.00}{\textbf{{#1}}}}
    \newcommand{\FunctionTok}[1]{\textcolor[rgb]{0.02,0.16,0.49}{{#1}}}
    \newcommand{\RegionMarkerTok}[1]{{#1}}
    \newcommand{\ErrorTok}[1]{\textcolor[rgb]{1.00,0.00,0.00}{\textbf{{#1}}}}
    \newcommand{\NormalTok}[1]{{#1}}
    
    % Additional commands for more recent versions of Pandoc
    \newcommand{\ConstantTok}[1]{\textcolor[rgb]{0.53,0.00,0.00}{{#1}}}
    \newcommand{\SpecialCharTok}[1]{\textcolor[rgb]{0.25,0.44,0.63}{{#1}}}
    \newcommand{\VerbatimStringTok}[1]{\textcolor[rgb]{0.25,0.44,0.63}{{#1}}}
    \newcommand{\SpecialStringTok}[1]{\textcolor[rgb]{0.73,0.40,0.53}{{#1}}}
    \newcommand{\ImportTok}[1]{{#1}}
    \newcommand{\DocumentationTok}[1]{\textcolor[rgb]{0.73,0.13,0.13}{\textit{{#1}}}}
    \newcommand{\AnnotationTok}[1]{\textcolor[rgb]{0.38,0.63,0.69}{\textbf{\textit{{#1}}}}}
    \newcommand{\CommentVarTok}[1]{\textcolor[rgb]{0.38,0.63,0.69}{\textbf{\textit{{#1}}}}}
    \newcommand{\VariableTok}[1]{\textcolor[rgb]{0.10,0.09,0.49}{{#1}}}
    \newcommand{\ControlFlowTok}[1]{\textcolor[rgb]{0.00,0.44,0.13}{\textbf{{#1}}}}
    \newcommand{\OperatorTok}[1]{\textcolor[rgb]{0.40,0.40,0.40}{{#1}}}
    \newcommand{\BuiltInTok}[1]{{#1}}
    \newcommand{\ExtensionTok}[1]{{#1}}
    \newcommand{\PreprocessorTok}[1]{\textcolor[rgb]{0.74,0.48,0.00}{{#1}}}
    \newcommand{\AttributeTok}[1]{\textcolor[rgb]{0.49,0.56,0.16}{{#1}}}
    \newcommand{\InformationTok}[1]{\textcolor[rgb]{0.38,0.63,0.69}{\textbf{\textit{{#1}}}}}
    \newcommand{\WarningTok}[1]{\textcolor[rgb]{0.38,0.63,0.69}{\textbf{\textit{{#1}}}}}
    
    
    % Define a nice break command that doesn't care if a line doesn't already
    % exist.
    \def\br{\hspace*{\fill} \\* }
    % Math Jax compatability definitions
    \def\gt{>}
    \def\lt{<}
    % Document parameters
    \title{lab04\_gene\_partial}
    
    
    

    % Pygments definitions
    
\makeatletter
\def\PY@reset{\let\PY@it=\relax \let\PY@bf=\relax%
    \let\PY@ul=\relax \let\PY@tc=\relax%
    \let\PY@bc=\relax \let\PY@ff=\relax}
\def\PY@tok#1{\csname PY@tok@#1\endcsname}
\def\PY@toks#1+{\ifx\relax#1\empty\else%
    \PY@tok{#1}\expandafter\PY@toks\fi}
\def\PY@do#1{\PY@bc{\PY@tc{\PY@ul{%
    \PY@it{\PY@bf{\PY@ff{#1}}}}}}}
\def\PY#1#2{\PY@reset\PY@toks#1+\relax+\PY@do{#2}}

\expandafter\def\csname PY@tok@w\endcsname{\def\PY@tc##1{\textcolor[rgb]{0.73,0.73,0.73}{##1}}}
\expandafter\def\csname PY@tok@c\endcsname{\let\PY@it=\textit\def\PY@tc##1{\textcolor[rgb]{0.25,0.50,0.50}{##1}}}
\expandafter\def\csname PY@tok@cp\endcsname{\def\PY@tc##1{\textcolor[rgb]{0.74,0.48,0.00}{##1}}}
\expandafter\def\csname PY@tok@k\endcsname{\let\PY@bf=\textbf\def\PY@tc##1{\textcolor[rgb]{0.00,0.50,0.00}{##1}}}
\expandafter\def\csname PY@tok@kp\endcsname{\def\PY@tc##1{\textcolor[rgb]{0.00,0.50,0.00}{##1}}}
\expandafter\def\csname PY@tok@kt\endcsname{\def\PY@tc##1{\textcolor[rgb]{0.69,0.00,0.25}{##1}}}
\expandafter\def\csname PY@tok@o\endcsname{\def\PY@tc##1{\textcolor[rgb]{0.40,0.40,0.40}{##1}}}
\expandafter\def\csname PY@tok@ow\endcsname{\let\PY@bf=\textbf\def\PY@tc##1{\textcolor[rgb]{0.67,0.13,1.00}{##1}}}
\expandafter\def\csname PY@tok@nb\endcsname{\def\PY@tc##1{\textcolor[rgb]{0.00,0.50,0.00}{##1}}}
\expandafter\def\csname PY@tok@nf\endcsname{\def\PY@tc##1{\textcolor[rgb]{0.00,0.00,1.00}{##1}}}
\expandafter\def\csname PY@tok@nc\endcsname{\let\PY@bf=\textbf\def\PY@tc##1{\textcolor[rgb]{0.00,0.00,1.00}{##1}}}
\expandafter\def\csname PY@tok@nn\endcsname{\let\PY@bf=\textbf\def\PY@tc##1{\textcolor[rgb]{0.00,0.00,1.00}{##1}}}
\expandafter\def\csname PY@tok@ne\endcsname{\let\PY@bf=\textbf\def\PY@tc##1{\textcolor[rgb]{0.82,0.25,0.23}{##1}}}
\expandafter\def\csname PY@tok@nv\endcsname{\def\PY@tc##1{\textcolor[rgb]{0.10,0.09,0.49}{##1}}}
\expandafter\def\csname PY@tok@no\endcsname{\def\PY@tc##1{\textcolor[rgb]{0.53,0.00,0.00}{##1}}}
\expandafter\def\csname PY@tok@nl\endcsname{\def\PY@tc##1{\textcolor[rgb]{0.63,0.63,0.00}{##1}}}
\expandafter\def\csname PY@tok@ni\endcsname{\let\PY@bf=\textbf\def\PY@tc##1{\textcolor[rgb]{0.60,0.60,0.60}{##1}}}
\expandafter\def\csname PY@tok@na\endcsname{\def\PY@tc##1{\textcolor[rgb]{0.49,0.56,0.16}{##1}}}
\expandafter\def\csname PY@tok@nt\endcsname{\let\PY@bf=\textbf\def\PY@tc##1{\textcolor[rgb]{0.00,0.50,0.00}{##1}}}
\expandafter\def\csname PY@tok@nd\endcsname{\def\PY@tc##1{\textcolor[rgb]{0.67,0.13,1.00}{##1}}}
\expandafter\def\csname PY@tok@s\endcsname{\def\PY@tc##1{\textcolor[rgb]{0.73,0.13,0.13}{##1}}}
\expandafter\def\csname PY@tok@sd\endcsname{\let\PY@it=\textit\def\PY@tc##1{\textcolor[rgb]{0.73,0.13,0.13}{##1}}}
\expandafter\def\csname PY@tok@si\endcsname{\let\PY@bf=\textbf\def\PY@tc##1{\textcolor[rgb]{0.73,0.40,0.53}{##1}}}
\expandafter\def\csname PY@tok@se\endcsname{\let\PY@bf=\textbf\def\PY@tc##1{\textcolor[rgb]{0.73,0.40,0.13}{##1}}}
\expandafter\def\csname PY@tok@sr\endcsname{\def\PY@tc##1{\textcolor[rgb]{0.73,0.40,0.53}{##1}}}
\expandafter\def\csname PY@tok@ss\endcsname{\def\PY@tc##1{\textcolor[rgb]{0.10,0.09,0.49}{##1}}}
\expandafter\def\csname PY@tok@sx\endcsname{\def\PY@tc##1{\textcolor[rgb]{0.00,0.50,0.00}{##1}}}
\expandafter\def\csname PY@tok@m\endcsname{\def\PY@tc##1{\textcolor[rgb]{0.40,0.40,0.40}{##1}}}
\expandafter\def\csname PY@tok@gh\endcsname{\let\PY@bf=\textbf\def\PY@tc##1{\textcolor[rgb]{0.00,0.00,0.50}{##1}}}
\expandafter\def\csname PY@tok@gu\endcsname{\let\PY@bf=\textbf\def\PY@tc##1{\textcolor[rgb]{0.50,0.00,0.50}{##1}}}
\expandafter\def\csname PY@tok@gd\endcsname{\def\PY@tc##1{\textcolor[rgb]{0.63,0.00,0.00}{##1}}}
\expandafter\def\csname PY@tok@gi\endcsname{\def\PY@tc##1{\textcolor[rgb]{0.00,0.63,0.00}{##1}}}
\expandafter\def\csname PY@tok@gr\endcsname{\def\PY@tc##1{\textcolor[rgb]{1.00,0.00,0.00}{##1}}}
\expandafter\def\csname PY@tok@ge\endcsname{\let\PY@it=\textit}
\expandafter\def\csname PY@tok@gs\endcsname{\let\PY@bf=\textbf}
\expandafter\def\csname PY@tok@gp\endcsname{\let\PY@bf=\textbf\def\PY@tc##1{\textcolor[rgb]{0.00,0.00,0.50}{##1}}}
\expandafter\def\csname PY@tok@go\endcsname{\def\PY@tc##1{\textcolor[rgb]{0.53,0.53,0.53}{##1}}}
\expandafter\def\csname PY@tok@gt\endcsname{\def\PY@tc##1{\textcolor[rgb]{0.00,0.27,0.87}{##1}}}
\expandafter\def\csname PY@tok@err\endcsname{\def\PY@bc##1{\setlength{\fboxsep}{0pt}\fcolorbox[rgb]{1.00,0.00,0.00}{1,1,1}{\strut ##1}}}
\expandafter\def\csname PY@tok@kc\endcsname{\let\PY@bf=\textbf\def\PY@tc##1{\textcolor[rgb]{0.00,0.50,0.00}{##1}}}
\expandafter\def\csname PY@tok@kd\endcsname{\let\PY@bf=\textbf\def\PY@tc##1{\textcolor[rgb]{0.00,0.50,0.00}{##1}}}
\expandafter\def\csname PY@tok@kn\endcsname{\let\PY@bf=\textbf\def\PY@tc##1{\textcolor[rgb]{0.00,0.50,0.00}{##1}}}
\expandafter\def\csname PY@tok@kr\endcsname{\let\PY@bf=\textbf\def\PY@tc##1{\textcolor[rgb]{0.00,0.50,0.00}{##1}}}
\expandafter\def\csname PY@tok@bp\endcsname{\def\PY@tc##1{\textcolor[rgb]{0.00,0.50,0.00}{##1}}}
\expandafter\def\csname PY@tok@fm\endcsname{\def\PY@tc##1{\textcolor[rgb]{0.00,0.00,1.00}{##1}}}
\expandafter\def\csname PY@tok@vc\endcsname{\def\PY@tc##1{\textcolor[rgb]{0.10,0.09,0.49}{##1}}}
\expandafter\def\csname PY@tok@vg\endcsname{\def\PY@tc##1{\textcolor[rgb]{0.10,0.09,0.49}{##1}}}
\expandafter\def\csname PY@tok@vi\endcsname{\def\PY@tc##1{\textcolor[rgb]{0.10,0.09,0.49}{##1}}}
\expandafter\def\csname PY@tok@vm\endcsname{\def\PY@tc##1{\textcolor[rgb]{0.10,0.09,0.49}{##1}}}
\expandafter\def\csname PY@tok@sa\endcsname{\def\PY@tc##1{\textcolor[rgb]{0.73,0.13,0.13}{##1}}}
\expandafter\def\csname PY@tok@sb\endcsname{\def\PY@tc##1{\textcolor[rgb]{0.73,0.13,0.13}{##1}}}
\expandafter\def\csname PY@tok@sc\endcsname{\def\PY@tc##1{\textcolor[rgb]{0.73,0.13,0.13}{##1}}}
\expandafter\def\csname PY@tok@dl\endcsname{\def\PY@tc##1{\textcolor[rgb]{0.73,0.13,0.13}{##1}}}
\expandafter\def\csname PY@tok@s2\endcsname{\def\PY@tc##1{\textcolor[rgb]{0.73,0.13,0.13}{##1}}}
\expandafter\def\csname PY@tok@sh\endcsname{\def\PY@tc##1{\textcolor[rgb]{0.73,0.13,0.13}{##1}}}
\expandafter\def\csname PY@tok@s1\endcsname{\def\PY@tc##1{\textcolor[rgb]{0.73,0.13,0.13}{##1}}}
\expandafter\def\csname PY@tok@mb\endcsname{\def\PY@tc##1{\textcolor[rgb]{0.40,0.40,0.40}{##1}}}
\expandafter\def\csname PY@tok@mf\endcsname{\def\PY@tc##1{\textcolor[rgb]{0.40,0.40,0.40}{##1}}}
\expandafter\def\csname PY@tok@mh\endcsname{\def\PY@tc##1{\textcolor[rgb]{0.40,0.40,0.40}{##1}}}
\expandafter\def\csname PY@tok@mi\endcsname{\def\PY@tc##1{\textcolor[rgb]{0.40,0.40,0.40}{##1}}}
\expandafter\def\csname PY@tok@il\endcsname{\def\PY@tc##1{\textcolor[rgb]{0.40,0.40,0.40}{##1}}}
\expandafter\def\csname PY@tok@mo\endcsname{\def\PY@tc##1{\textcolor[rgb]{0.40,0.40,0.40}{##1}}}
\expandafter\def\csname PY@tok@ch\endcsname{\let\PY@it=\textit\def\PY@tc##1{\textcolor[rgb]{0.25,0.50,0.50}{##1}}}
\expandafter\def\csname PY@tok@cm\endcsname{\let\PY@it=\textit\def\PY@tc##1{\textcolor[rgb]{0.25,0.50,0.50}{##1}}}
\expandafter\def\csname PY@tok@cpf\endcsname{\let\PY@it=\textit\def\PY@tc##1{\textcolor[rgb]{0.25,0.50,0.50}{##1}}}
\expandafter\def\csname PY@tok@c1\endcsname{\let\PY@it=\textit\def\PY@tc##1{\textcolor[rgb]{0.25,0.50,0.50}{##1}}}
\expandafter\def\csname PY@tok@cs\endcsname{\let\PY@it=\textit\def\PY@tc##1{\textcolor[rgb]{0.25,0.50,0.50}{##1}}}

\def\PYZbs{\char`\\}
\def\PYZus{\char`\_}
\def\PYZob{\char`\{}
\def\PYZcb{\char`\}}
\def\PYZca{\char`\^}
\def\PYZam{\char`\&}
\def\PYZlt{\char`\<}
\def\PYZgt{\char`\>}
\def\PYZsh{\char`\#}
\def\PYZpc{\char`\%}
\def\PYZdl{\char`\$}
\def\PYZhy{\char`\-}
\def\PYZsq{\char`\'}
\def\PYZdq{\char`\"}
\def\PYZti{\char`\~}
% for compatibility with earlier versions
\def\PYZat{@}
\def\PYZlb{[}
\def\PYZrb{]}
\makeatother


    % Exact colors from NB
    \definecolor{incolor}{rgb}{0.0, 0.0, 0.5}
    \definecolor{outcolor}{rgb}{0.545, 0.0, 0.0}



    
    % Prevent overflowing lines due to hard-to-break entities
    \sloppy 
    % Setup hyperref package
    \hypersetup{
      breaklinks=true,  % so long urls are correctly broken across lines
      colorlinks=true,
      urlcolor=urlcolor,
      linkcolor=linkcolor,
      citecolor=citecolor,
      }
    % Slightly bigger margins than the latex defaults
    
    \geometry{verbose,tmargin=1in,bmargin=1in,lmargin=1in,rmargin=1in}
    
    

    \begin{document}
    
    
    \maketitle
    
    

    
    \hypertarget{lab-logistic-regression-for-gene-expression-data}{%
\section{Lab: Logistic Regression for Gene Expression
Data}\label{lab-logistic-regression-for-gene-expression-data}}

In this lab, we use logistic regression to predict biological
characteristics (``phenotypes'') from gene expression data. In addition
to the concepts in \href{./demo04_breast_cancer.ipynb}{breast cancer
demo}, you will learn to: * Handle missing data * Perform binary
classification, and evaluating performance using various metrics *
Perform multi-class logistic classification, and evaluating performance
using accuracy and confusion matrix * Use L1-regularization to promote
sparse weights for improved estimation (Grad students only)

\hypertarget{background}{%
\subsection{Background}\label{background}}

Genes are the basic unit in the DNA and encode blueprints for proteins.
When proteins are synthesized from a gene, the gene is said to
``express''. Micro-arrays are devices that measure the expression levels
of large numbers of genes in parallel. By finding correlations between
expression levels and phenotypes, scientists can identify possible
genetic markers for biological characteristics.

The data in this lab comes from:

https://archive.ics.uci.edu/ml/datasets/Mice+Protein+Expression

In this data, mice were characterized by three properties: * Whether
they had down's syndrome (trisomy) or not * Whether they were stimulated
to learn or not * Whether they had a drug memantine or a saline control
solution.

With these three choices, there are 8 possible classes for each mouse.
For each mouse, the expression levels were measured across 77 genes. We
will see if the characteristics can be predicted from the gene
expression levels. This classification could reveal which genes are
potentially involved in Down's syndrome and if drugs and learning have
any noticeable effects.

    \hypertarget{load-the-data}{%
\subsection{Load the Data}\label{load-the-data}}

We begin by loading the standard modules.

    \begin{Verbatim}[commandchars=\\\{\}]
{\color{incolor}In [{\color{incolor}1}]:} \PY{k+kn}{import} \PY{n+nn}{pandas} \PY{k}{as} \PY{n+nn}{pd}
        \PY{k+kn}{import} \PY{n+nn}{numpy} \PY{k}{as} \PY{n+nn}{np}
        \PY{k+kn}{import} \PY{n+nn}{matplotlib}
        \PY{k+kn}{import} \PY{n+nn}{matplotlib}\PY{n+nn}{.}\PY{n+nn}{pyplot} \PY{k}{as} \PY{n+nn}{plt}
        \PY{o}{\PYZpc{}}\PY{k}{matplotlib} inline
        \PY{k+kn}{from} \PY{n+nn}{sklearn} \PY{k}{import} \PY{n}{linear\PYZus{}model}\PY{p}{,} \PY{n}{preprocessing}
\end{Verbatim}


    Use the \texttt{pd.read\_excel} command to read the data from

https://archive.ics.uci.edu/ml/machine-learning-databases/00342/Data\_Cortex\_Nuclear.xls

into a dataframe \texttt{df}. Use the \texttt{index\_col} option to
specify that column 0 is the index. Use the \texttt{df.head()} to print
the first few rows.

    \begin{Verbatim}[commandchars=\\\{\}]
{\color{incolor}In [{\color{incolor}2}]:} \PY{c+c1}{\PYZsh{} TODO}
        \PY{n}{df} \PY{o}{=} \PY{n}{pd}\PY{o}{.}\PY{n}{read\PYZus{}excel}\PY{p}{(}\PY{l+s+s1}{\PYZsq{}}\PY{l+s+s1}{https://archive.ics.uci.edu/ml/machine\PYZhy{}learning\PYZhy{}databases/00342/Data\PYZus{}Cortex\PYZus{}Nuclear.xls}\PY{l+s+s1}{\PYZsq{}}\PY{p}{,} \PY{n}{index\PYZus{}col}\PY{o}{=}\PY{l+m+mi}{0}\PY{p}{)}
        \PY{n}{df}\PY{o}{.}\PY{n}{head}\PY{p}{(}\PY{p}{)}
\end{Verbatim}


\begin{Verbatim}[commandchars=\\\{\}]
{\color{outcolor}Out[{\color{outcolor}2}]:}          DYRK1A\_N   ITSN1\_N    BDNF\_N     NR1\_N    NR2A\_N    pAKT\_N   pBRAF\_N  \textbackslash{}
        MouseID                                                                         
        309\_1    0.503644  0.747193  0.430175  2.816329  5.990152  0.218830  0.177565   
        309\_2    0.514617  0.689064  0.411770  2.789514  5.685038  0.211636  0.172817   
        309\_3    0.509183  0.730247  0.418309  2.687201  5.622059  0.209011  0.175722   
        309\_4    0.442107  0.617076  0.358626  2.466947  4.979503  0.222886  0.176463   
        309\_5    0.434940  0.617430  0.358802  2.365785  4.718679  0.213106  0.173627   
        
                 pCAMKII\_N   pCREB\_N    pELK\_N   {\ldots}     pCFOS\_N     SYP\_N  H3AcK18\_N  \textbackslash{}
        MouseID                                  {\ldots}                                    
        309\_1     2.373744  0.232224  1.750936   {\ldots}    0.108336  0.427099   0.114783   
        309\_2     2.292150  0.226972  1.596377   {\ldots}    0.104315  0.441581   0.111974   
        309\_3     2.283337  0.230247  1.561316   {\ldots}    0.106219  0.435777   0.111883   
        309\_4     2.152301  0.207004  1.595086   {\ldots}    0.111262  0.391691   0.130405   
        309\_5     2.134014  0.192158  1.504230   {\ldots}    0.110694  0.434154   0.118481   
        
                   EGR1\_N  H3MeK4\_N    CaNA\_N  Genotype  Treatment  Behavior   class  
        MouseID                                                                       
        309\_1    0.131790  0.128186  1.675652   Control  Memantine       C/S  c-CS-m  
        309\_2    0.135103  0.131119  1.743610   Control  Memantine       C/S  c-CS-m  
        309\_3    0.133362  0.127431  1.926427   Control  Memantine       C/S  c-CS-m  
        309\_4    0.147444  0.146901  1.700563   Control  Memantine       C/S  c-CS-m  
        309\_5    0.140314  0.148380  1.839730   Control  Memantine       C/S  c-CS-m  
        
        [5 rows x 81 columns]
\end{Verbatim}
            
    This data has missing values. The site:

http://pandas.pydata.org/pandas-docs/stable/missing\_data.html

has an excellent summary of methods to deal with missing values.
Following the techniques there, create a new data frame \texttt{df1}
where the missing values in each column are filled with the mean values
from the non-missing values.

    \begin{Verbatim}[commandchars=\\\{\}]
{\color{incolor}In [{\color{incolor}3}]:} \PY{c+c1}{\PYZsh{} TODO}
        \PY{n}{df1} \PY{o}{=} \PY{n}{df}\PY{o}{.}\PY{n}{fillna}\PY{p}{(}\PY{n}{df}\PY{o}{.}\PY{n}{mean}\PY{p}{(}\PY{p}{)}\PY{p}{)}
\end{Verbatim}


    \hypertarget{binary-classification-for-downs-syndrome}{%
\subsection{Binary Classification for Down's
Syndrome}\label{binary-classification-for-downs-syndrome}}

We will first predict the binary class label in
\texttt{df1{[}\textquotesingle{}Genotype\textquotesingle{}{]}} which
indicates if the mouse has Down's syndrome or not. Get the string values
in \texttt{df1{[}\textquotesingle{}Genotype\textquotesingle{}{]}.values}
and convert this to a numeric vector \texttt{y} with 0 or 1. You may
wish to use the \texttt{np.unique} command with the
\texttt{return\_inverse=True} option.

    \begin{Verbatim}[commandchars=\\\{\}]
{\color{incolor}In [{\color{incolor}4}]:} \PY{c+c1}{\PYZsh{} TODO}
        \PY{n}{y} \PY{o}{=} \PY{n}{np}\PY{o}{.}\PY{n}{unique}\PY{p}{(}\PY{n}{df1}\PY{p}{[}\PY{l+s+s1}{\PYZsq{}}\PY{l+s+s1}{Genotype}\PY{l+s+s1}{\PYZsq{}}\PY{p}{]}\PY{o}{.}\PY{n}{values}\PY{p}{,} \PY{n}{return\PYZus{}inverse} \PY{o}{=} \PY{k+kc}{True}\PY{p}{)}\PY{p}{[}\PY{l+m+mi}{1}\PY{p}{]}
\end{Verbatim}


    As predictors, get all but the last four columns of the dataframes.
Standardize the data matrix and call the standardized matrix
\texttt{Xs}. The predictors are the expression levels of the 77 genes.

    \begin{Verbatim}[commandchars=\\\{\}]
{\color{incolor}In [{\color{incolor}5}]:} \PY{c+c1}{\PYZsh{} TODO}
        \PY{n}{Xs} \PY{o}{=} \PY{n}{preprocessing}\PY{o}{.}\PY{n}{scale}\PY{p}{(}\PY{n}{np}\PY{o}{.}\PY{n}{array}\PY{p}{(}\PY{n}{df1}\PY{p}{[}\PY{n}{df1}\PY{o}{.}\PY{n}{columns}\PY{p}{[}\PY{p}{:}\PY{o}{\PYZhy{}}\PY{l+m+mi}{4}\PY{p}{]}\PY{p}{]}\PY{p}{)}\PY{p}{)}
\end{Verbatim}


    Create a \texttt{LogisticRegression} object \texttt{logreg} and
\texttt{fit} the training data. Use \texttt{C\ =\ 1e5}.

    \begin{Verbatim}[commandchars=\\\{\}]
{\color{incolor}In [{\color{incolor}6}]:} \PY{c+c1}{\PYZsh{} TODO}
        \PY{n}{logreg} \PY{o}{=} \PY{n}{linear\PYZus{}model}\PY{o}{.}\PY{n}{LogisticRegression}\PY{p}{(}\PY{n}{C}\PY{o}{=}\PY{l+m+mf}{1e5}\PY{p}{)}
        \PY{n}{logreg}\PY{o}{.}\PY{n}{fit}\PY{p}{(}\PY{n}{Xs}\PY{p}{,}\PY{n}{y}\PY{p}{)}
\end{Verbatim}


\begin{Verbatim}[commandchars=\\\{\}]
{\color{outcolor}Out[{\color{outcolor}6}]:} LogisticRegression(C=100000.0, class\_weight=None, dual=False,
                  fit\_intercept=True, intercept\_scaling=1, max\_iter=100,
                  multi\_class='ovr', n\_jobs=1, penalty='l2', random\_state=None,
                  solver='liblinear', tol=0.0001, verbose=0, warm\_start=False)
\end{Verbatim}
            
    Measure the accuracy of the classifer. That is, use the
\texttt{logreg.predict} function to predict labels \texttt{yhat} and
measure the fraction of time that the predictions match the true labels.
Also, plot the ROC curve, and measure the AUC. Later, we will properly
measure the accuracy and AUC on cross-validation data.

    \begin{Verbatim}[commandchars=\\\{\}]
{\color{incolor}In [{\color{incolor}7}]:} \PY{c+c1}{\PYZsh{} TODO}
        \PY{n}{yhat} \PY{o}{=} \PY{n}{logreg}\PY{o}{.}\PY{n}{predict}\PY{p}{(}\PY{n}{Xs}\PY{p}{)}
        \PY{n}{yprob} \PY{o}{=} \PY{n}{logreg}\PY{o}{.}\PY{n}{predict\PYZus{}proba}\PY{p}{(}\PY{n}{Xs}\PY{p}{)}
        \PY{n+nb}{print}\PY{p}{(}\PY{l+s+s2}{\PYZdq{}}\PY{l+s+s2}{Accuracy of the classifier = }\PY{l+s+si}{\PYZpc{}f}\PY{l+s+s2}{\PYZdq{}} \PY{o}{\PYZpc{}} \PY{n}{np}\PY{o}{.}\PY{n}{mean}\PY{p}{(}\PY{n}{yhat} \PY{o}{==} \PY{n}{y}\PY{p}{)}\PY{p}{)}
        \PY{k+kn}{from} \PY{n+nn}{sklearn} \PY{k}{import} \PY{n}{metrics}
        \PY{n}{fpr}\PY{p}{,} \PY{n}{tpr}\PY{p}{,} \PY{n}{thresholds} \PY{o}{=} \PY{n}{metrics}\PY{o}{.}\PY{n}{roc\PYZus{}curve}\PY{p}{(}\PY{n}{y}\PY{p}{,} \PY{n}{yprob}\PY{p}{[}\PY{p}{:}\PY{p}{,}\PY{l+m+mi}{1}\PY{p}{]}\PY{p}{)}
        
        \PY{n}{plt}\PY{o}{.}\PY{n}{plot}\PY{p}{(}\PY{n}{fpr}\PY{p}{,} \PY{n}{tpr}\PY{p}{)}
        \PY{n}{plt}\PY{o}{.}\PY{n}{grid}\PY{p}{(}\PY{p}{)}
        \PY{n}{plt}\PY{o}{.}\PY{n}{xlabel}\PY{p}{(}\PY{l+s+s1}{\PYZsq{}}\PY{l+s+s1}{FPR}\PY{l+s+s1}{\PYZsq{}}\PY{p}{)}
        \PY{n}{plt}\PY{o}{.}\PY{n}{ylabel}\PY{p}{(}\PY{l+s+s1}{\PYZsq{}}\PY{l+s+s1}{TPR}\PY{l+s+s1}{\PYZsq{}}\PY{p}{)}
        \PY{n}{plt}\PY{o}{.}\PY{n}{show}\PY{p}{(}\PY{p}{)}
        
        \PY{n}{auc}\PY{o}{=}\PY{n}{metrics}\PY{o}{.}\PY{n}{roc\PYZus{}auc\PYZus{}score}\PY{p}{(}\PY{n}{y}\PY{p}{,} \PY{n}{yprob}\PY{p}{[}\PY{p}{:}\PY{p}{,}\PY{l+m+mi}{1}\PY{p}{]}\PY{p}{)}
        \PY{n+nb}{print}\PY{p}{(}\PY{l+s+s2}{\PYZdq{}}\PY{l+s+s2}{AUC = }\PY{l+s+si}{\PYZpc{}f}\PY{l+s+s2}{\PYZdq{}} \PY{o}{\PYZpc{}} \PY{n}{auc}\PY{p}{)}
\end{Verbatim}


    \begin{Verbatim}[commandchars=\\\{\}]
Accuracy of the classifier = 1.000000

    \end{Verbatim}

    \begin{center}
    \adjustimage{max size={0.9\linewidth}{0.9\paperheight}}{output_14_1.png}
    \end{center}
    { \hspace*{\fill} \\}
    
    \begin{Verbatim}[commandchars=\\\{\}]
AUC = 1.000000

    \end{Verbatim}

    \hypertarget{interpreting-the-weight-vector}{%
\subsection{Interpreting the weight
vector}\label{interpreting-the-weight-vector}}

    Create a stem plot of the coefficients, \texttt{W} in the logistic
regression model. You can get the coefficients from
\texttt{logreg.coef\_}, but you will need to reshape this to a 1D array.

    \begin{Verbatim}[commandchars=\\\{\}]
{\color{incolor}In [{\color{incolor}8}]:} \PY{c+c1}{\PYZsh{} TODO}
        \PY{n}{W} \PY{o}{=} \PY{n}{logreg}\PY{o}{.}\PY{n}{coef\PYZus{}}
        \PY{n}{plt}\PY{o}{.}\PY{n}{stem}\PY{p}{(}\PY{n}{logreg}\PY{o}{.}\PY{n}{coef\PYZus{}}\PY{o}{.}\PY{n}{reshape}\PY{p}{(}\PY{o}{\PYZhy{}}\PY{l+m+mi}{1}\PY{p}{,}\PY{l+m+mi}{1}\PY{p}{)}\PY{p}{)}
        \PY{n}{plt}\PY{o}{.}\PY{n}{xlabel}\PY{p}{(}\PY{l+s+s1}{\PYZsq{}}\PY{l+s+s1}{i}\PY{l+s+s1}{\PYZsq{}}\PY{p}{)}
        \PY{n}{plt}\PY{o}{.}\PY{n}{ylabel}\PY{p}{(}\PY{l+s+s1}{\PYZsq{}}\PY{l+s+s1}{Weight}\PY{l+s+s1}{\PYZsq{}}\PY{p}{)}
        \PY{n}{plt}\PY{o}{.}\PY{n}{show}\PY{p}{(}\PY{p}{)}
\end{Verbatim}


    \begin{center}
    \adjustimage{max size={0.9\linewidth}{0.9\paperheight}}{output_17_0.png}
    \end{center}
    { \hspace*{\fill} \\}
    
    You should see that \texttt{W{[}i{]}} is very large for a few components
\texttt{i}. These are the genes that are likely to be most involved in
Down's Syndrome.

Find the names of the genes for two components \texttt{i} where the
magnitude of \texttt{W{[}i{]}} is largest.

    \begin{Verbatim}[commandchars=\\\{\}]
{\color{incolor}In [{\color{incolor}9}]:} \PY{c+c1}{\PYZsh{} TODO}
        \PY{n+nb}{print}\PY{p}{(}\PY{l+s+s1}{\PYZsq{}}\PY{l+s+s1}{The two components are }\PY{l+s+se}{\PYZbs{}\PYZsq{}}\PY{l+s+si}{\PYZpc{}s}\PY{l+s+se}{\PYZbs{}\PYZsq{}}\PY{l+s+s1}{ and }\PY{l+s+se}{\PYZbs{}\PYZsq{}}\PY{l+s+si}{\PYZpc{}s}\PY{l+s+se}{\PYZbs{}\PYZsq{}}\PY{l+s+s1}{.}\PY{l+s+s1}{\PYZsq{}} \PY{o}{\PYZpc{}} \PY{n+nb}{tuple}\PY{p}{(}\PY{n}{df1}\PY{o}{.}\PY{n}{columns}\PY{p}{[}\PY{n}{np}\PY{o}{.}\PY{n}{argsort}\PY{p}{(}\PY{o}{\PYZhy{}}\PY{n}{logreg}\PY{o}{.}\PY{n}{coef\PYZus{}}\PY{p}{)}\PY{p}{[}\PY{p}{:}\PY{p}{,}\PY{p}{:}\PY{l+m+mi}{2}\PY{p}{]}\PY{p}{]}\PY{p}{[}\PY{l+m+mi}{0}\PY{p}{]}\PY{p}{)}\PY{p}{)}
\end{Verbatim}


    \begin{Verbatim}[commandchars=\\\{\}]
The two components are 'ITSN1\_N' and 'TIAM1\_N'.

    \end{Verbatim}

    \hypertarget{cross-validation}{%
\subsection{Cross Validation}\label{cross-validation}}

The above meaured the accuracy on the training data. It is more accurate
to measure the accuracy on the test data. Perform 10-fold cross
validation and measure the average precision, recall and f1-score, as
well as the AUC. Note, that in performing the cross-validation, you will
want to randomly permute the test and training sets using the
\texttt{shuffle} option. In this data set, all the samples from each
class are bunched together, so shuffling is essential. Print the mean
precision, recall and f1-score and error rate across all the folds.

    \begin{Verbatim}[commandchars=\\\{\}]
{\color{incolor}In [{\color{incolor}10}]:} \PY{c+c1}{\PYZsh{} TODO}
         \PY{k+kn}{from} \PY{n+nn}{sklearn}\PY{n+nn}{.}\PY{n+nn}{model\PYZus{}selection} \PY{k}{import} \PY{n}{KFold}
         \PY{k+kn}{from} \PY{n+nn}{sklearn}\PY{n+nn}{.}\PY{n+nn}{metrics} \PY{k}{import} \PY{n}{precision\PYZus{}recall\PYZus{}fscore\PYZus{}support}
         
         \PY{n}{nfold} \PY{o}{=} \PY{l+m+mi}{10}
         \PY{n}{kf} \PY{o}{=} \PY{n}{KFold}\PY{p}{(}\PY{n}{n\PYZus{}splits}\PY{o}{=}\PY{n}{nfold}\PY{p}{)}
         
         \PY{n}{prec} \PY{o}{=} \PY{p}{[}\PY{p}{]}
         \PY{n}{rec} \PY{o}{=} \PY{p}{[}\PY{p}{]}
         \PY{n}{f1} \PY{o}{=} \PY{p}{[}\PY{p}{]}
         \PY{n}{err} \PY{o}{=} \PY{p}{[}\PY{p}{]}
         
         \PY{k}{for} \PY{n}{train}\PY{p}{,} \PY{n}{test} \PY{o+ow}{in} \PY{n}{kf}\PY{o}{.}\PY{n}{split}\PY{p}{(}\PY{n}{Xs}\PY{p}{)}\PY{p}{:}            
             \PY{c+c1}{\PYZsh{} Get training and test data}
             \PY{n}{Xtr} \PY{o}{=} \PY{n}{Xs}\PY{p}{[}\PY{n}{train}\PY{p}{,}\PY{p}{:}\PY{p}{]}
             \PY{n}{ytr} \PY{o}{=} \PY{n}{y}\PY{p}{[}\PY{n}{train}\PY{p}{]}
             \PY{n}{Xts} \PY{o}{=} \PY{n}{Xs}\PY{p}{[}\PY{n}{test}\PY{p}{,}\PY{p}{:}\PY{p}{]}
             \PY{n}{yts} \PY{o}{=} \PY{n}{y}\PY{p}{[}\PY{n}{test}\PY{p}{]}
             
             \PY{c+c1}{\PYZsh{} Fit a model}
             \PY{n}{logreg}\PY{o}{.}\PY{n}{fit}\PY{p}{(}\PY{n}{Xtr}\PY{p}{,} \PY{n}{ytr}\PY{p}{)}
             \PY{n}{yhat} \PY{o}{=} \PY{n}{logreg}\PY{o}{.}\PY{n}{predict}\PY{p}{(}\PY{n}{Xts}\PY{p}{)}
             
             \PY{c+c1}{\PYZsh{} Measure performance}
             \PY{n}{preci}\PY{p}{,}\PY{n}{reci}\PY{p}{,}\PY{n}{f1i}\PY{p}{,}\PY{n}{\PYZus{}}\PY{o}{=} \PY{n}{precision\PYZus{}recall\PYZus{}fscore\PYZus{}support}\PY{p}{(}\PY{n}{yts}\PY{p}{,} \PY{n}{yhat}\PY{p}{,} \PY{n}{average}\PY{o}{=}\PY{l+s+s1}{\PYZsq{}}\PY{l+s+s1}{binary}\PY{l+s+s1}{\PYZsq{}}\PY{p}{)} 
             \PY{n}{prec}\PY{o}{.}\PY{n}{append}\PY{p}{(}\PY{n}{preci}\PY{p}{)}
             \PY{n}{rec}\PY{o}{.}\PY{n}{append}\PY{p}{(}\PY{n}{reci}\PY{p}{)}
             \PY{n}{f1}\PY{o}{.}\PY{n}{append}\PY{p}{(}\PY{n}{f1i}\PY{p}{)}
             \PY{n}{erri} \PY{o}{=} \PY{n}{np}\PY{o}{.}\PY{n}{mean}\PY{p}{(}\PY{n}{yhat} \PY{o}{!=} \PY{n}{yts}\PY{p}{)}
             \PY{n}{err}\PY{o}{.}\PY{n}{append}\PY{p}{(}\PY{n}{erri}\PY{p}{)}
         
         \PY{c+c1}{\PYZsh{} Take average values of the metrics}
         \PY{n}{precm} \PY{o}{=} \PY{n}{np}\PY{o}{.}\PY{n}{mean}\PY{p}{(}\PY{n}{prec}\PY{p}{)}
         \PY{n}{recm} \PY{o}{=} \PY{n}{np}\PY{o}{.}\PY{n}{mean}\PY{p}{(}\PY{n}{rec}\PY{p}{)}
         \PY{n}{f1m} \PY{o}{=} \PY{n}{np}\PY{o}{.}\PY{n}{mean}\PY{p}{(}\PY{n}{f1}\PY{p}{)}
         \PY{n}{errm} \PY{o}{=} \PY{n}{np}\PY{o}{.}\PY{n}{mean}\PY{p}{(}\PY{n}{err}\PY{p}{)}
         
         \PY{c+c1}{\PYZsh{} Compute the standard errors}
         \PY{n}{prec\PYZus{}se} \PY{o}{=} \PY{n}{np}\PY{o}{.}\PY{n}{std}\PY{p}{(}\PY{n}{prec}\PY{p}{)}\PY{o}{/}\PY{n}{np}\PY{o}{.}\PY{n}{sqrt}\PY{p}{(}\PY{n}{nfold}\PY{o}{\PYZhy{}}\PY{l+m+mi}{1}\PY{p}{)}
         \PY{n}{rec\PYZus{}se} \PY{o}{=} \PY{n}{np}\PY{o}{.}\PY{n}{std}\PY{p}{(}\PY{n}{rec}\PY{p}{)}\PY{o}{/}\PY{n}{np}\PY{o}{.}\PY{n}{sqrt}\PY{p}{(}\PY{n}{nfold}\PY{o}{\PYZhy{}}\PY{l+m+mi}{1}\PY{p}{)}
         \PY{n}{f1\PYZus{}se} \PY{o}{=} \PY{n}{np}\PY{o}{.}\PY{n}{std}\PY{p}{(}\PY{n}{f1}\PY{p}{)}\PY{o}{/}\PY{n}{np}\PY{o}{.}\PY{n}{sqrt}\PY{p}{(}\PY{n}{nfold}\PY{o}{\PYZhy{}}\PY{l+m+mi}{1}\PY{p}{)}
         \PY{n}{err\PYZus{}se} \PY{o}{=} \PY{n}{np}\PY{o}{.}\PY{n}{std}\PY{p}{(}\PY{n}{err}\PY{p}{)}\PY{o}{/}\PY{n}{np}\PY{o}{.}\PY{n}{sqrt}\PY{p}{(}\PY{n}{nfold}\PY{o}{\PYZhy{}}\PY{l+m+mi}{1}\PY{p}{)}
         
         \PY{n+nb}{print}\PY{p}{(}\PY{l+s+s1}{\PYZsq{}}\PY{l+s+s1}{Precision =  }\PY{l+s+si}{\PYZpc{}0.4f}\PY{l+s+s1}{, SE = }\PY{l+s+si}{\PYZpc{}0.4f}\PY{l+s+s1}{\PYZsq{}} \PY{o}{\PYZpc{}} \PY{p}{(}\PY{n}{precm}\PY{p}{,}\PY{n}{prec\PYZus{}se}\PY{p}{)}\PY{p}{)}
         \PY{n+nb}{print}\PY{p}{(}\PY{l+s+s1}{\PYZsq{}}\PY{l+s+s1}{Recall =     }\PY{l+s+si}{\PYZpc{}0.4f}\PY{l+s+s1}{, SE = }\PY{l+s+si}{\PYZpc{}0.4f}\PY{l+s+s1}{\PYZsq{}} \PY{o}{\PYZpc{}} \PY{p}{(}\PY{n}{recm}\PY{p}{,} \PY{n}{rec\PYZus{}se}\PY{p}{)}\PY{p}{)}
         \PY{n+nb}{print}\PY{p}{(}\PY{l+s+s1}{\PYZsq{}}\PY{l+s+s1}{f1 =         }\PY{l+s+si}{\PYZpc{}0.4f}\PY{l+s+s1}{, SE = }\PY{l+s+si}{\PYZpc{}0.4f}\PY{l+s+s1}{\PYZsq{}} \PY{o}{\PYZpc{}} \PY{p}{(}\PY{n}{f1m}\PY{p}{,} \PY{n}{f1\PYZus{}se}\PY{p}{)}\PY{p}{)}
         \PY{n+nb}{print}\PY{p}{(}\PY{l+s+s1}{\PYZsq{}}\PY{l+s+s1}{Error rate = }\PY{l+s+si}{\PYZpc{}0.4f}\PY{l+s+s1}{, SE = }\PY{l+s+si}{\PYZpc{}0.4f}\PY{l+s+s1}{\PYZsq{}} \PY{o}{\PYZpc{}} \PY{p}{(}\PY{n}{errm}\PY{p}{,} \PY{n}{err\PYZus{}se}\PY{p}{)}\PY{p}{)}
\end{Verbatim}


    \begin{Verbatim}[commandchars=\\\{\}]
/usr/local/lib/python3.6/site-packages/sklearn/metrics/classification.py:1137: UndefinedMetricWarning: Recall and F-score are ill-defined and being set to 0.0 due to no true samples.
  'recall', 'true', average, warn\_for)

    \end{Verbatim}

    \begin{Verbatim}[commandchars=\\\{\}]
Precision =  0.4689, SE = 0.1590
Recall =     0.3378, SE = 0.1211
f1 =         0.3804, SE = 0.1308
Error rate = 0.3565, SE = 0.0576

    \end{Verbatim}

    \hypertarget{multi-class-classification}{%
\subsection{Multi-Class
Classification}\label{multi-class-classification}}

Now use the response variable in
\texttt{df1{[}\textquotesingle{}class\textquotesingle{}{]}}. This has 8
possible classes. Use the \texttt{np.unique} funtion as before to
convert this to a vector \texttt{y} with values 0 to 7.

    \begin{Verbatim}[commandchars=\\\{\}]
{\color{incolor}In [{\color{incolor}11}]:} \PY{c+c1}{\PYZsh{} TODO}
         \PY{n}{y} \PY{o}{=} \PY{n}{np}\PY{o}{.}\PY{n}{unique}\PY{p}{(}\PY{n}{df1}\PY{p}{[}\PY{l+s+s1}{\PYZsq{}}\PY{l+s+s1}{class}\PY{l+s+s1}{\PYZsq{}}\PY{p}{]}\PY{o}{.}\PY{n}{values}\PY{p}{,} \PY{n}{return\PYZus{}inverse} \PY{o}{=} \PY{k+kc}{True}\PY{p}{)}\PY{p}{[}\PY{l+m+mi}{1}\PY{p}{]}
\end{Verbatim}


    Fit a multi-class logistic model by creating a
\texttt{LogisticRegression} object, \texttt{logreg} and then calling the
\texttt{logreg.fit} method. In general, you could either use the `one
over rest (ovr)' option or the `multinomial' option. In this exercise
use the default `ovr' and \texttt{C=1}. As an optional exercise, you
could also compare the results obtained with these two options.

    \begin{Verbatim}[commandchars=\\\{\}]
{\color{incolor}In [{\color{incolor}12}]:} \PY{c+c1}{\PYZsh{} TODO}
         \PY{n}{logreg} \PY{o}{=} \PY{n}{linear\PYZus{}model}\PY{o}{.}\PY{n}{LogisticRegression}\PY{p}{(}\PY{n}{multi\PYZus{}class}\PY{o}{=}\PY{l+s+s1}{\PYZsq{}}\PY{l+s+s1}{ovr}\PY{l+s+s1}{\PYZsq{}}\PY{p}{,} \PY{n}{C}\PY{o}{=}\PY{l+m+mi}{1}\PY{p}{)}
         \PY{n}{logreg}\PY{o}{.}\PY{n}{fit}\PY{p}{(}\PY{n}{Xs}\PY{p}{,} \PY{n}{y}\PY{p}{)}
\end{Verbatim}


\begin{Verbatim}[commandchars=\\\{\}]
{\color{outcolor}Out[{\color{outcolor}12}]:} LogisticRegression(C=1, class\_weight=None, dual=False, fit\_intercept=True,
                   intercept\_scaling=1, max\_iter=100, multi\_class='ovr', n\_jobs=1,
                   penalty='l2', random\_state=None, solver='liblinear', tol=0.0001,
                   verbose=0, warm\_start=False)
\end{Verbatim}
            
    Measure the accuracy on the training data.

    \begin{Verbatim}[commandchars=\\\{\}]
{\color{incolor}In [{\color{incolor}13}]:} \PY{c+c1}{\PYZsh{} TODO}
         \PY{n}{yhat} \PY{o}{=} \PY{n}{logreg}\PY{o}{.}\PY{n}{predict}\PY{p}{(}\PY{n}{Xs}\PY{p}{)}
         \PY{n+nb}{print}\PY{p}{(}\PY{l+s+s2}{\PYZdq{}}\PY{l+s+s2}{Accuracy on training data = }\PY{l+s+si}{\PYZpc{}f}\PY{l+s+s2}{\PYZdq{}} \PY{o}{\PYZpc{}} \PY{n}{np}\PY{o}{.}\PY{n}{mean}\PY{p}{(}\PY{n}{yhat} \PY{o}{==} \PY{n}{y}\PY{p}{)}\PY{p}{)}
\end{Verbatim}


    \begin{Verbatim}[commandchars=\\\{\}]
Accuracy on training data = 0.999074

    \end{Verbatim}

    Now perform 10-fold cross validation, and measure the confusion matrix
\texttt{C} on the test data in each fold. You can use the
\texttt{confustion\_matrix} method in the \texttt{sklearn} package. Add
the confusion matrix counts across all folds and then normalize the rows
of the confusion matrix so that they sum to one. Thus, each element
\texttt{C{[}i,j{]}} will represent the fraction of samples where
\texttt{yhat==j} given \texttt{ytrue==i}. Print the confusion matrix.
You can use the command

\begin{verbatim}
print(np.array_str(C, precision=4, suppress_small=True))
\end{verbatim}

to create a nicely formatted print. Also print the overall mean and SE
of the test accuracy across the folds.

    \begin{Verbatim}[commandchars=\\\{\}]
{\color{incolor}In [{\color{incolor}14}]:} \PY{k+kn}{from} \PY{n+nn}{sklearn}\PY{n+nn}{.}\PY{n+nn}{metrics} \PY{k}{import} \PY{n}{confusion\PYZus{}matrix}
         \PY{k+kn}{from} \PY{n+nn}{sklearn}\PY{n+nn}{.}\PY{n+nn}{model\PYZus{}selection} \PY{k}{import} \PY{n}{KFold}
         
         \PY{c+c1}{\PYZsh{} TODO}
         \PY{n}{C} \PY{o}{=} \PY{n}{np}\PY{o}{.}\PY{n}{zeros}\PY{p}{(}\PY{p}{(}\PY{l+m+mi}{8}\PY{p}{,} \PY{l+m+mi}{8}\PY{p}{)}\PY{p}{)}
         \PY{n}{nfold} \PY{o}{=} \PY{l+m+mi}{10}
         \PY{n}{kf} \PY{o}{=} \PY{n}{KFold}\PY{p}{(}\PY{n}{n\PYZus{}splits}\PY{o}{=}\PY{n}{nfold}\PY{p}{,} \PY{n}{shuffle}\PY{o}{=}\PY{k+kc}{True}\PY{p}{)}
         
         \PY{k}{for} \PY{n}{train}\PY{p}{,} \PY{n}{test} \PY{o+ow}{in} \PY{n}{kf}\PY{o}{.}\PY{n}{split}\PY{p}{(}\PY{n}{Xs}\PY{p}{)}\PY{p}{:}
             \PY{n}{Xtr} \PY{o}{=} \PY{n}{Xs}\PY{p}{[}\PY{n}{train}\PY{p}{,}\PY{p}{:}\PY{p}{]}
             \PY{n}{ytr} \PY{o}{=} \PY{n}{y}\PY{p}{[}\PY{n}{train}\PY{p}{]}
             \PY{n}{Xts} \PY{o}{=} \PY{n}{Xs}\PY{p}{[}\PY{n}{test}\PY{p}{,}\PY{p}{:}\PY{p}{]}
             \PY{n}{yts} \PY{o}{=} \PY{n}{y}\PY{p}{[}\PY{n}{test}\PY{p}{]}
             
             \PY{n}{logreg}\PY{o}{.}\PY{n}{fit}\PY{p}{(}\PY{n}{Xtr}\PY{p}{,} \PY{n}{ytr}\PY{p}{)}
             \PY{n}{yhat} \PY{o}{=} \PY{n}{logreg}\PY{o}{.}\PY{n}{predict}\PY{p}{(}\PY{n}{Xts}\PY{p}{)}
             
             \PY{n}{C} \PY{o}{+}\PY{o}{=} \PY{n}{confusion\PYZus{}matrix}\PY{p}{(}\PY{n}{yts}\PY{p}{,} \PY{n}{yhat}\PY{p}{)}
             
             \PY{n}{RSS} \PY{o}{=} \PY{n}{np}\PY{o}{.}\PY{n}{mean}\PY{p}{(}\PY{p}{(}\PY{n}{yhat}\PY{o}{\PYZhy{}}\PY{n}{yts}\PY{p}{)}\PY{o}{*}\PY{o}{*}\PY{l+m+mi}{2}\PY{p}{)}\PY{o}{/}\PY{n}{np}\PY{o}{.}\PY{n}{std}\PY{p}{(}\PY{n}{yts}\PY{p}{)}\PY{o}{*}\PY{o}{*}\PY{l+m+mi}{2}
             \PY{n+nb}{print}\PY{p}{(}\PY{l+s+s2}{\PYZdq{}}\PY{l+s+s2}{Test error rate for fold }\PY{l+s+si}{\PYZpc{}02d}\PY{l+s+s2}{ = }\PY{l+s+si}{\PYZpc{}.4f}\PY{l+s+s2}{\PYZdq{}} \PY{o}{\PYZpc{}} \PY{p}{(}\PY{l+m+mi}{11}\PY{o}{\PYZhy{}}\PY{n}{nfold}\PY{p}{,} \PY{n}{RSS}\PY{p}{)}\PY{p}{)}
             \PY{n}{nfold} \PY{o}{\PYZhy{}}\PY{o}{=} \PY{l+m+mi}{1}
             
         \PY{n}{new\PYZus{}matrix} \PY{o}{=} \PY{n}{preprocessing}\PY{o}{.}\PY{n}{normalize}\PY{p}{(}\PY{n}{C}\PY{p}{,} \PY{n}{axis}\PY{o}{=}\PY{l+m+mi}{1}\PY{p}{)}
         
         \PY{n+nb}{print}\PY{p}{(}\PY{l+s+s2}{\PYZdq{}}\PY{l+s+se}{\PYZbs{}n}\PY{l+s+s2}{Confusion Matrix: }\PY{l+s+se}{\PYZbs{}n}\PY{l+s+s2}{\PYZdq{}}\PY{p}{,} \PY{n}{np}\PY{o}{.}\PY{n}{array\PYZus{}str}\PY{p}{(}\PY{n}{new\PYZus{}matrix}\PY{p}{,} \PY{n}{precision}\PY{o}{=}\PY{l+m+mi}{4}\PY{p}{,} \PY{n}{suppress\PYZus{}small}\PY{o}{=}\PY{k+kc}{True}\PY{p}{)}\PY{p}{)}
\end{Verbatim}


    \begin{Verbatim}[commandchars=\\\{\}]
Test error rate for fold 01 = 0.0349
Test error rate for fold 02 = 0.0190
Test error rate for fold 03 = 0.0542
Test error rate for fold 04 = 0.0856
Test error rate for fold 05 = 0.0346
Test error rate for fold 06 = 0.0343
Test error rate for fold 07 = 0.0326
Test error rate for fold 08 = 0.0000
Test error rate for fold 09 = 0.0000
Test error rate for fold 10 = 0.0000

Confusion Matrix: 
 [[0.9997 0.0069 0.0069 0.     0.0207 0.     0.     0.    ]
 [0.0076 0.9999 0.     0.     0.0076 0.0076 0.     0.    ]
 [0.     0.     1.     0.     0.     0.     0.     0.0067]
 [0.0075 0.     0.     1.     0.     0.     0.     0.    ]
 [0.0076 0.0229 0.     0.     0.9997 0.     0.     0.    ]
 [0.     0.     0.     0.     0.     1.     0.     0.    ]
 [0.     0.     0.0075 0.     0.     0.     1.     0.    ]
 [0.     0.     0.     0.     0.     0.     0.     1.    ]]

    \end{Verbatim}

    Re-run the logistic regression on the entire training data and get the
weight coefficients. This should be a 8 x 77 matrix. Create a stem plot
of the first row of this matrix to see the coefficients on each of the
genes for the first class.

    \begin{Verbatim}[commandchars=\\\{\}]
{\color{incolor}In [{\color{incolor}15}]:} \PY{c+c1}{\PYZsh{} TODO}
         \PY{n}{logreg}\PY{o}{.}\PY{n}{fit}\PY{p}{(}\PY{n}{Xs}\PY{p}{,} \PY{n}{y}\PY{p}{)}
         \PY{n}{W} \PY{o}{=} \PY{n}{logreg}\PY{o}{.}\PY{n}{coef\PYZus{}}
         \PY{n}{plt}\PY{o}{.}\PY{n}{stem}\PY{p}{(}\PY{n}{logreg}\PY{o}{.}\PY{n}{coef\PYZus{}}\PY{p}{[}\PY{l+m+mi}{0}\PY{p}{]}\PY{p}{)}
         \PY{n}{plt}\PY{o}{.}\PY{n}{show}\PY{p}{(}\PY{p}{)}
\end{Verbatim}


    \begin{center}
    \adjustimage{max size={0.9\linewidth}{0.9\paperheight}}{output_31_0.png}
    \end{center}
    { \hspace*{\fill} \\}
    
    \hypertarget{l1-regularization}{%
\subsection{L1-Regularization}\label{l1-regularization}}

Graduate students only complete this section.

In most genetic problems, only a limited number of the tested genes are
likely influence any particular attribute. Hence, we would expect that
the weight coefficients in the logistic regression model should be
sparse. That is, they should be zero on any gene that plays no role in
the particular attribute of interest. Genetic analysis commonly imposes
sparsity by adding an l1-penalty term. Read the \texttt{sklearn}
\href{http://scikit-learn.org/stable/modules/generated/sklearn.linear_model.LogisticRegression.html}{documentation}
on the \texttt{LogisticRegression} class to see how to set the
l1-penalty and the inverse regularization strength, \texttt{C}.

Using the model selection strategies from the
\href{../unit03_model_sel/demo03_2_prostate.ipynb}{prostate cancer
analysis demo}, use K-fold cross validation to select an appropriate
inverse regularization strength.\\
* Use 10-fold cross validation * You should select around 20 values of
\texttt{C}. It is up to you to find a good range. * For each C and each
fold, you should compute the classification error rate * For each C and
each fold, you should also determine the nubmer of non-zero coefficients
for the first class. For this purpse, you can assume coefficient with
magnitude \textless{}0.01 as zero.

    \begin{Verbatim}[commandchars=\\\{\}]
{\color{incolor}In [{\color{incolor}16}]:} \PY{c+c1}{\PYZsh{} TODO}
         \PY{n}{nfold} \PY{o}{=} \PY{l+m+mi}{10}
         \PY{n}{kf} \PY{o}{=} \PY{n}{KFold}\PY{p}{(}\PY{n}{n\PYZus{}splits}\PY{o}{=}\PY{n}{nfold}\PY{p}{,} \PY{n}{shuffle}\PY{o}{=}\PY{k+kc}{True}\PY{p}{)}
         
         \PY{n}{npen} \PY{o}{=} \PY{l+m+mi}{20}
         \PY{n}{C\PYZus{}test} \PY{o}{=} \PY{n}{np}\PY{o}{.}\PY{n}{logspace}\PY{p}{(}\PY{o}{\PYZhy{}}\PY{l+m+mi}{2}\PY{p}{,}\PY{l+m+mi}{2}\PY{p}{,}\PY{n}{npen}\PY{p}{)}
         \PY{n}{err\PYZus{}rate} \PY{o}{=} \PY{n}{np}\PY{o}{.}\PY{n}{zeros}\PY{p}{(}\PY{p}{(}\PY{n}{npen}\PY{p}{,}\PY{n}{nfold}\PY{p}{)}\PY{p}{)}
         \PY{n}{num\PYZus{}nonzerocoef} \PY{o}{=} \PY{n}{np}\PY{o}{.}\PY{n}{zeros}\PY{p}{(}\PY{p}{(}\PY{n}{npen}\PY{p}{,}\PY{n}{nfold}\PY{p}{)}\PY{p}{)}
         
         \PY{c+c1}{\PYZsh{} Create the logistic regression object}
         \PY{n}{logreg} \PY{o}{=} \PY{n}{linear\PYZus{}model}\PY{o}{.}\PY{n}{LogisticRegression}\PY{p}{(}\PY{n}{penalty}\PY{o}{=}\PY{l+s+s1}{\PYZsq{}}\PY{l+s+s1}{l1}\PY{l+s+s1}{\PYZsq{}}\PY{p}{,} \PY{n}{warm\PYZus{}start}\PY{o}{=}\PY{k+kc}{True}\PY{p}{)}
         
         \PY{k}{for} \PY{n}{ifold}\PY{p}{,} \PY{n}{Ind} \PY{o+ow}{in} \PY{n+nb}{enumerate}\PY{p}{(}\PY{n}{kf}\PY{o}{.}\PY{n}{split}\PY{p}{(}\PY{n}{Xs}\PY{p}{)}\PY{p}{)}\PY{p}{:} 
             \PY{c+c1}{\PYZsh{} Get training and test data}
             \PY{n}{Itr}\PY{p}{,} \PY{n}{Its} \PY{o}{=} \PY{n}{Ind}
             \PY{n}{Xtr} \PY{o}{=} \PY{n}{Xs}\PY{p}{[}\PY{n}{Itr}\PY{p}{,}\PY{p}{:}\PY{p}{]}
             \PY{n}{ytr} \PY{o}{=} \PY{n}{y}\PY{p}{[}\PY{n}{Itr}\PY{p}{]}
             \PY{n}{Xts} \PY{o}{=} \PY{n}{Xs}\PY{p}{[}\PY{n}{Its}\PY{p}{,}\PY{p}{:}\PY{p}{]}
             \PY{n}{yts} \PY{o}{=} \PY{n}{y}\PY{p}{[}\PY{n}{Its}\PY{p}{]}
             
             \PY{c+c1}{\PYZsh{} Loop over penalty levels}
             \PY{k}{for} \PY{n}{ipen}\PY{p}{,} \PY{n}{c} \PY{o+ow}{in} \PY{n+nb}{enumerate}\PY{p}{(}\PY{n}{C\PYZus{}test}\PY{p}{)}\PY{p}{:}
                 
                 \PY{c+c1}{\PYZsh{} Set the penalty level        }
                 \PY{n}{logreg}\PY{o}{.}\PY{n}{C} \PY{o}{=} \PY{n}{c}
             
                 \PY{c+c1}{\PYZsh{} Fit a model on the training data}
                 \PY{n}{logreg}\PY{o}{.}\PY{n}{fit}\PY{p}{(}\PY{n}{Xtr}\PY{p}{,} \PY{n}{ytr}\PY{p}{)}
             
                 \PY{c+c1}{\PYZsh{} Predict the labels on the test set.}
                 \PY{n}{yhat} \PY{o}{=} \PY{n}{logreg}\PY{o}{.}\PY{n}{predict}\PY{p}{(}\PY{n}{Xts}\PY{p}{)}
                 
                 \PY{c+c1}{\PYZsh{} Measure the accuracy}
                 \PY{n}{err\PYZus{}rate}\PY{p}{[}\PY{n}{ipen}\PY{p}{,} \PY{n}{ifold}\PY{p}{]} \PY{o}{=} \PY{n}{np}\PY{o}{.}\PY{n}{mean}\PY{p}{(}\PY{n}{yhat} \PY{o}{!=} \PY{n}{yts}\PY{p}{)}
                 \PY{n}{num\PYZus{}nonzerocoef}\PY{p}{[}\PY{n}{ipen}\PY{p}{,} \PY{n}{ifold}\PY{p}{]}\PY{o}{=}\PY{n}{np}\PY{o}{.}\PY{n}{sum}\PY{p}{(}\PY{n+nb}{abs}\PY{p}{(}\PY{n}{logreg}\PY{o}{.}\PY{n}{coef\PYZus{}}\PY{p}{)} \PY{o}{\PYZgt{}} \PY{l+m+mf}{0.01}\PY{p}{)}
             \PY{n+nb}{print}\PY{p}{(}\PY{l+s+s2}{\PYZdq{}}\PY{l+s+s2}{Fold }\PY{l+s+si}{\PYZpc{}d}\PY{l+s+s2}{\PYZdq{}} \PY{o}{\PYZpc{}} \PY{p}{(}\PY{n}{ifold}\PY{o}{+}\PY{l+m+mi}{1}\PY{p}{)}\PY{p}{)}
\end{Verbatim}


    \begin{Verbatim}[commandchars=\\\{\}]
Fold 1
Fold 2
Fold 3
Fold 4
Fold 5
Fold 6
Fold 7
Fold 8
Fold 9
Fold 10

    \end{Verbatim}

    Now compute the mean and standard error on the error rate for each
\texttt{C} and plot the results (Use \texttt{errorbar()} method). Also
determine and print the minimum test error rate and corresponding C
value.

    \begin{Verbatim}[commandchars=\\\{\}]
{\color{incolor}In [{\color{incolor}17}]:} \PY{c+c1}{\PYZsh{} TODO}
         \PY{n}{err\PYZus{}mean} \PY{o}{=} \PY{n}{np}\PY{o}{.}\PY{n}{mean}\PY{p}{(}\PY{n}{err\PYZus{}rate}\PY{p}{,} \PY{n}{axis}\PY{o}{=}\PY{l+m+mi}{1}\PY{p}{)}
         \PY{n}{num\PYZus{}nonzerocoef\PYZus{}mean} \PY{o}{=} \PY{n}{np}\PY{o}{.}\PY{n}{mean}\PY{p}{(}\PY{n}{num\PYZus{}nonzerocoef}\PY{p}{,} \PY{n}{axis}\PY{o}{=}\PY{l+m+mi}{1}\PY{p}{)}
         \PY{n}{err\PYZus{}se} \PY{o}{=} \PY{n}{np}\PY{o}{.}\PY{n}{std}\PY{p}{(}\PY{n}{err\PYZus{}rate}\PY{p}{,} \PY{n}{axis}\PY{o}{=}\PY{l+m+mi}{1}\PY{p}{)} \PY{o}{/} \PY{n}{np}\PY{o}{.}\PY{n}{sqrt}\PY{p}{(}\PY{n}{nfold}\PY{o}{\PYZhy{}}\PY{l+m+mi}{1}\PY{p}{)}
         \PY{n}{plt}\PY{o}{.}\PY{n}{errorbar}\PY{p}{(}\PY{n}{np}\PY{o}{.}\PY{n}{log10}\PY{p}{(}\PY{n}{C\PYZus{}test}\PY{p}{)}\PY{p}{,} \PY{n}{err\PYZus{}mean}\PY{p}{,} \PY{n}{marker}\PY{o}{=}\PY{l+s+s1}{\PYZsq{}}\PY{l+s+s1}{o}\PY{l+s+s1}{\PYZsq{}}\PY{p}{,}\PY{n}{yerr}\PY{o}{=}\PY{n}{err\PYZus{}se}\PY{p}{)}
         \PY{n}{plt}\PY{o}{.}\PY{n}{grid}\PY{p}{(}\PY{p}{)}
         \PY{n}{plt}\PY{o}{.}\PY{n}{xlabel}\PY{p}{(}\PY{l+s+sa}{r}\PY{l+s+s1}{\PYZsq{}}\PY{l+s+s1}{\PYZdl{}}\PY{l+s+s1}{\PYZbs{}}\PY{l+s+s1}{log\PYZus{}}\PY{l+s+si}{\PYZob{}10\PYZcb{}}\PY{l+s+s1}{(C)\PYZdl{}}\PY{l+s+s1}{\PYZsq{}}\PY{p}{)}
         \PY{n}{plt}\PY{o}{.}\PY{n}{ylabel}\PY{p}{(}\PY{l+s+s1}{\PYZsq{}}\PY{l+s+s1}{Error rate}\PY{l+s+s1}{\PYZsq{}}\PY{p}{)}
         
         \PY{n}{imin} \PY{o}{=} \PY{n}{np}\PY{o}{.}\PY{n}{argmin}\PY{p}{(}\PY{n}{err\PYZus{}mean}\PY{p}{)}
         
         \PY{n+nb}{print}\PY{p}{(}\PY{l+s+s2}{\PYZdq{}}\PY{l+s+s2}{The minimum test error rate = }\PY{l+s+si}{\PYZpc{}0.4e}\PY{l+s+s2}{, SE=}\PY{l+s+si}{\PYZpc{}0.4e}\PY{l+s+s2}{\PYZdq{}} \PY{o}{\PYZpc{}} \PY{p}{(}\PY{n}{err\PYZus{}mean}\PY{p}{[}\PY{n}{imin}\PY{p}{]}\PY{p}{,} \PY{n}{err\PYZus{}se}\PY{p}{[}\PY{n}{imin}\PY{p}{]}\PY{p}{)}\PY{p}{)}
         \PY{n+nb}{print}\PY{p}{(}\PY{l+s+s2}{\PYZdq{}}\PY{l+s+s2}{The C value corresponding to minimum error = }\PY{l+s+si}{\PYZpc{}0.4e}\PY{l+s+s2}{\PYZdq{}} \PY{o}{\PYZpc{}} \PY{p}{(}\PY{n}{C\PYZus{}test}\PY{p}{[}\PY{n}{imin}\PY{p}{]}\PY{p}{)}\PY{p}{)}
\end{Verbatim}


    \begin{Verbatim}[commandchars=\\\{\}]
The minimum test error rate = 8.3333e-03, SE=2.9117e-03
The C value corresponding to minimum error = 7.8476e-01

    \end{Verbatim}

    \begin{center}
    \adjustimage{max size={0.9\linewidth}{0.9\paperheight}}{output_35_1.png}
    \end{center}
    { \hspace*{\fill} \\}
    
    We see that the minimum error rate is significantly below the classifier
that did not use the l1-penalty. Use the one-standard error rule to
determine the optimal \texttt{C} and the corresponding test error rate.
Note that because \texttt{C} is inversely proportional to the
regularization strength, you want to select a \texttt{C} as \emph{small}
as possible while meeting the error target!

    \begin{Verbatim}[commandchars=\\\{\}]
{\color{incolor}In [{\color{incolor}18}]:} \PY{c+c1}{\PYZsh{} TODO}
         \PY{n}{err\PYZus{}tgt} \PY{o}{=} \PY{n}{err\PYZus{}mean}\PY{p}{[}\PY{n}{imin}\PY{p}{]} \PY{o}{+} \PY{n}{err\PYZus{}se}\PY{p}{[}\PY{n}{imin}\PY{p}{]}
         \PY{n}{iopt} \PY{o}{=} \PY{n}{np}\PY{o}{.}\PY{n}{where}\PY{p}{(}\PY{n}{err\PYZus{}mean} \PY{o}{\PYZlt{}} \PY{n}{err\PYZus{}tgt}\PY{p}{)}\PY{p}{[}\PY{l+m+mi}{0}\PY{p}{]}\PY{p}{[}\PY{l+m+mi}{0}\PY{p}{]}
         \PY{n}{C\PYZus{}opt} \PY{o}{=} \PY{n}{C\PYZus{}test}\PY{p}{[}\PY{n}{iopt}\PY{p}{]}
         
         \PY{n+nb}{print}\PY{p}{(}\PY{l+s+s2}{\PYZdq{}}\PY{l+s+s2}{Optimal C = }\PY{l+s+si}{\PYZpc{}0.4e}\PY{l+s+s2}{\PYZdq{}} \PY{o}{\PYZpc{}} \PY{n}{C\PYZus{}opt}\PY{p}{)}
         \PY{n+nb}{print}\PY{p}{(}\PY{l+s+s2}{\PYZdq{}}\PY{l+s+s2}{The test error rate = }\PY{l+s+si}{\PYZpc{}0.4e}\PY{l+s+s2}{, SE=}\PY{l+s+si}{\PYZpc{}0.4e}\PY{l+s+s2}{\PYZdq{}} \PY{o}{\PYZpc{}} \PY{p}{(}\PY{n}{err\PYZus{}mean}\PY{p}{[}\PY{n}{iopt}\PY{p}{]}\PY{p}{,} \PY{n}{err\PYZus{}se}\PY{p}{[}\PY{n}{iopt}\PY{p}{]}\PY{p}{)}\PY{p}{)}
         \PY{n+nb}{print}\PY{p}{(}\PY{l+s+s1}{\PYZsq{}}\PY{l+s+s1}{Accuracy = }\PY{l+s+si}{\PYZpc{}0.4f}\PY{l+s+s1}{, SE=}\PY{l+s+si}{\PYZpc{}0.4f}\PY{l+s+s1}{\PYZsq{}} \PY{o}{\PYZpc{}} \PY{p}{(}\PY{l+m+mi}{1}\PY{o}{\PYZhy{}}\PY{n}{err\PYZus{}mean}\PY{p}{[}\PY{n}{iopt}\PY{p}{]}\PY{p}{,} \PY{n}{err\PYZus{}se}\PY{p}{[}\PY{n}{iopt}\PY{p}{]}\PY{p}{)}\PY{p}{)}
\end{Verbatim}


    \begin{Verbatim}[commandchars=\\\{\}]
Optimal C = 4.8329e-01
The test error rate = 1.1111e-02, SE=2.3097e-03
Accuracy = 0.9889, SE=0.0023

    \end{Verbatim}

    \textbf{Question:} How does the test error rate compare with the
classifier that did not use the l1-penalty? Explain why.

    \textbf{Type Answer Here:}

The error rate with L1-penalty is a bit smaller than the classifier that
did not use the L1-penalty. I think the reason is that L1-penalty shrink
some of the features to be zero, which means these features play no role
on the prediction. So the prediction will focus on the true related
features. Thus, it leads to a smaller error rate.

    Now plot the nubmer of non-zero coefficients for the first class for
different C values. Also determine and print the number of non-zero
coefficients corresponding to C\_opt.

    \begin{Verbatim}[commandchars=\\\{\}]
{\color{incolor}In [{\color{incolor}19}]:} \PY{c+c1}{\PYZsh{} TODO}
         \PY{n}{num\PYZus{}nonzerocoef\PYZus{}mean} \PY{o}{=} \PY{n}{np}\PY{o}{.}\PY{n}{mean}\PY{p}{(}\PY{n}{num\PYZus{}nonzerocoef}\PY{p}{,} \PY{n}{axis}\PY{o}{=}\PY{l+m+mi}{1}\PY{p}{)}
         \PY{n}{plt}\PY{o}{.}\PY{n}{plot}\PY{p}{(}\PY{n}{np}\PY{o}{.}\PY{n}{log10}\PY{p}{(}\PY{n}{C\PYZus{}test}\PY{p}{)}\PY{p}{,}\PY{n}{num\PYZus{}nonzerocoef\PYZus{}mean}\PY{p}{)}
         
         \PY{n}{plt}\PY{o}{.}\PY{n}{grid}\PY{p}{(}\PY{p}{)}
         \PY{n}{plt}\PY{o}{.}\PY{n}{xlabel}\PY{p}{(}\PY{l+s+s1}{\PYZsq{}}\PY{l+s+s1}{log10(C)}\PY{l+s+s1}{\PYZsq{}}\PY{p}{)}
         \PY{n}{plt}\PY{o}{.}\PY{n}{ylabel}\PY{p}{(}\PY{l+s+s1}{\PYZsq{}}\PY{l+s+s1}{Num of nonzero coeff.}\PY{l+s+s1}{\PYZsq{}}\PY{p}{)}
         
         \PY{n+nb}{print}\PY{p}{(}\PY{l+s+s2}{\PYZdq{}}\PY{l+s+s2}{The number of non\PYZhy{}zero coefficients for the optimal C = }\PY{l+s+si}{\PYZpc{}f}\PY{l+s+s2}{\PYZdq{}} \PY{o}{\PYZpc{}} \PY{n}{num\PYZus{}nonzerocoef\PYZus{}mean}\PY{p}{[}\PY{n}{iopt}\PY{p}{]}\PY{p}{)}
\end{Verbatim}


    \begin{Verbatim}[commandchars=\\\{\}]
The number of non-zero coefficients for the optimal C = 249.300000

    \end{Verbatim}

    \begin{center}
    \adjustimage{max size={0.9\linewidth}{0.9\paperheight}}{output_41_1.png}
    \end{center}
    { \hspace*{\fill} \\}
    
    For the optimal \texttt{C}, fit the model on the entire training data
with l1 regularization. Find the resulting weight matrix,
\texttt{W\_l1}. Plot the first row of this weight matrix and compare it
to the first row of the weight matrix without the regularization. You
should see that, with l1-regularization, the weight matrix is much more
sparse and hence the roles of particular genes are more clearly visible.
Please also compare the accuracy for the training data using optimal
\texttt{C} with the previous results not using LASSO regularization. Do
you expect the accuracy to improve?

    \begin{Verbatim}[commandchars=\\\{\}]
{\color{incolor}In [{\color{incolor}20}]:} \PY{c+c1}{\PYZsh{} TODO}
         \PY{n}{logreg} \PY{o}{=} \PY{n}{linear\PYZus{}model}\PY{o}{.}\PY{n}{LogisticRegression}\PY{p}{(}\PY{n}{C}\PY{o}{=}\PY{n}{C\PYZus{}opt}\PY{p}{,}\PY{n}{penalty}\PY{o}{=}\PY{l+s+s1}{\PYZsq{}}\PY{l+s+s1}{l1}\PY{l+s+s1}{\PYZsq{}}\PY{p}{)}
         \PY{n}{logreg}\PY{o}{.}\PY{n}{C}\PY{o}{=} \PY{n}{C\PYZus{}opt}
         \PY{n}{logreg}\PY{o}{.}\PY{n}{fit}\PY{p}{(}\PY{n}{Xs}\PY{p}{,}\PY{n}{y}\PY{p}{)}
         \PY{n}{yhat} \PY{o}{=} \PY{n}{logreg}\PY{o}{.}\PY{n}{predict}\PY{p}{(}\PY{n}{Xs}\PY{p}{)}
         \PY{n}{acc} \PY{o}{=} \PY{n}{np}\PY{o}{.}\PY{n}{mean}\PY{p}{(}\PY{n}{yhat} \PY{o}{==} \PY{n}{y}\PY{p}{)}
         \PY{n+nb}{print}\PY{p}{(}\PY{l+s+s1}{\PYZsq{}}\PY{l+s+s1}{Accuracy on the training data is }\PY{l+s+si}{\PYZob{}0:f\PYZcb{}}\PY{l+s+s1}{\PYZsq{}}\PY{o}{.}\PY{n}{format}\PY{p}{(}\PY{n}{acc}\PY{p}{)}\PY{p}{)}
         \PY{n}{W\PYZus{}l1} \PY{o}{=} \PY{n}{logreg}\PY{o}{.}\PY{n}{coef\PYZus{}}
         
         \PY{n}{plt}\PY{o}{.}\PY{n}{figure}\PY{p}{(}\PY{n}{figsize}\PY{o}{=}\PY{p}{(}\PY{l+m+mi}{7}\PY{p}{,}\PY{l+m+mi}{7}\PY{p}{)}\PY{p}{)}
         \PY{n}{plt}\PY{o}{.}\PY{n}{subplot}\PY{p}{(}\PY{l+m+mi}{2}\PY{p}{,}\PY{l+m+mi}{1}\PY{p}{,}\PY{l+m+mi}{1}\PY{p}{)}
         \PY{n}{plt}\PY{o}{.}\PY{n}{stem}\PY{p}{(}\PY{n}{W}\PY{p}{[}\PY{l+m+mi}{0}\PY{p}{,}\PY{p}{:}\PY{p}{]}\PY{p}{)}
         \PY{n}{plt}\PY{o}{.}\PY{n}{title}\PY{p}{(}\PY{l+s+s1}{\PYZsq{}}\PY{l+s+s1}{No regularization}\PY{l+s+s1}{\PYZsq{}}\PY{p}{)}
         \PY{n}{plt}\PY{o}{.}\PY{n}{subplot}\PY{p}{(}\PY{l+m+mi}{2}\PY{p}{,}\PY{l+m+mi}{1}\PY{p}{,}\PY{l+m+mi}{2}\PY{p}{)}
         \PY{n}{plt}\PY{o}{.}\PY{n}{stem}\PY{p}{(}\PY{n}{W\PYZus{}l1}\PY{p}{[}\PY{l+m+mi}{0}\PY{p}{,}\PY{p}{:}\PY{p}{]}\PY{p}{)}
         \PY{n}{plt}\PY{o}{.}\PY{n}{title}\PY{p}{(}\PY{l+s+s1}{\PYZsq{}}\PY{l+s+s1}{l1\PYZhy{}regularization}\PY{l+s+s1}{\PYZsq{}}\PY{p}{)}
         \PY{n}{plt}\PY{o}{.}\PY{n}{show}\PY{p}{(}\PY{p}{)}
\end{Verbatim}


    \begin{Verbatim}[commandchars=\\\{\}]
Accuracy on the training data is 0.997222

    \end{Verbatim}

    \begin{center}
    \adjustimage{max size={0.9\linewidth}{0.9\paperheight}}{output_43_1.png}
    \end{center}
    { \hspace*{\fill} \\}
    

    % Add a bibliography block to the postdoc
    
    
    
    \end{document}
